%!TEX root = ../main.tex
\chapter{Asymptotic Properties}
There's a large and somewhat technical literature on the asymptotic properties of different model selection procedures, and we won't have time to do it justice here.
Leeb and P\"{o}tscher (2009a) give an overview.
For more details, in line with the presentation given below, see Sin and White (1992, 1996), P\"{o}tscher (1991), and Leeb and P\"{o}tscher (2005).
To learn more about the tradeoff between consistency and efficiency, see Yang (2005, 2007).

\section{Introduction}
Up until now we've made proceeded by setting forth desiderata for model selection, e.g. minimize the KL divergence or predictive mean-squared error, and then making enough assumptions until we could derive a criterion. 
And although the details of the derivations were all different, in each of the examples we've considered to far, the result amounted to adding a penalty to the maximized log-likelihood to account for model complexity, for example:
	\begin{eqnarray*}
		% TIC &=& 2\ell_T(\widehat{\theta}) -\mbox{trace}\left\{\widehat{J}^{-1} \widehat{K} \right\}\\
		AIC &=& 2\ell_T(\widehat{\theta}) - 2\; \mbox{length}(\theta)\\
		BIC &=& 2\ell_T(\widehat{\theta}) - \log(T)\; \mbox{length}(\theta)
	\end{eqnarray*}
We're now going to take a completely different perspective. 
Instead of asking what assumptions we need to derive a particular criterion, we'll ask ``given the penalty term that this criterion applies to the log-likelihood, how will it perform in large samples?'' 
We'll concern ourselves in particular with two properties: \textbf{consistency} and \textbf{efficiency}. 

\paragraph{Consistency} 
Suppose that we have a set of candidate models, one of which is actually the true DGP.
It seems clear that in this setting we'd like our model selection procedure to correctly identify the true DGP as the sample size grows.
This is the idea behind consistency.
Traditionally, we say that a model selection criterion is \textbf{consistent} if it selects the true DGP with probability approaching one as $T\rightarrow \infty$.
Since this notion only makes sense if the set of candidate models contains the true DGP, a fairly strong assumption, we can also consider slightly different notions of consistency as in Sin and White (1992, 1996).
We will explore this below.
The crucial point about consistent selection is that, in the limit, the probability that the ``best'' model is chosen approaches one.
Which model this is, of course, depends on how we have defined best.


\paragraph{Efficiency} 
It's somewhat rare that the goal of model selection is to determine which model is the ``truth'' or even which model is the KL minimizer. 
More commonly we estimate a model for \emph{some specific purpose}: perhaps we want to estimate a particular parameter or make a good forecast. 
From this perspective it is natural to look for a model selection criterion that with good risk properties, e.g.\ low mean-squared error.
Intuitively, we'l like the criterion to perform ``almost as well'' as the risk-optimal model in our candidate set. 
This property, which we'll make more precise below, is called \textbf{efficiency}. 

\paragraph{Efficiency or Consistency: Pick One}
You may be thinking ``consistency and efficiency both sound great so let's find a criterion that satisfies them both!''
Unfortunately, this turns out to be impossible: if a model selection criterion is consistent it cannot be efficient, and vice-versa. 
For more on this point, see Yang (2005, 2007).
Perhaps a more informative way of putting this is that there is an unavoidably price to be paid for consistent model selection in terms of poor risk properties.
We will see an illustration of this below in a simple example based on estimating the mean of a normal population.


\section{Penalizing the Log-Likelihood}
Suppose we want to choose the model that minimizes the KL-divergence, as described above in the chapter on AIC-type criteria.
Will our model selection criteria give us the ``right answer'' in the limit as we obtain more and more data?
To answer this question we will distinguish two properties, which Sin and White (1992, 1996) call ``weak consistency'' and ``consistency.''
This terminology is a little confusing, since we usually encounter a distinction between weak and \emph{strong} consistency that corresponds to whether we have convergence in probability or convergence almost surely.
That's \emph{not} the distinction that is being drawn here.
Instead the point is to distinguish between what happens when two or more models ``tie'' for lowest KL-divergence in the population.
The property of \emph{weak consistency} requires only that we never select a model that \emph{does not} minimize the KL-divergence as the sample size goes to infinity.
In contrast \emph{consistency} requires that we select the model with the fewest parameters among all those that minimize the KL-divergence.
A criterion that is not consistent but is weakly consistent is sometimes called \textbf{conservative}.

\paragraph{Setup:}
Let $g$ be the true, unknown data density. 
Now consider a collection of models $M_k$ indexed by $k = 1, 2, \hdots, K$ where $\theta_k$ is the parameter vector under model $M_k$ and $\widehat{\theta_k}$ is the corresponding maximum likelihood estimator.
Let $f_{k,t}(y_t|\theta_k)$ be the density of observation $t$ under model $k$ and suppose we're interested in choosing a model to mininimze the KL divergence from $g$ to $f_k$.
For simplicity, suppose that we can express the likelihood of model $k$ as $\sum_{t=1}^T \log f_{k,t}(Y_t| \theta_k)$. 
Note: we do \emph{not} assume the data are independent. 


\paragraph{General Form of Information Criteria}
We will consider model selection criteria for the following form:
	$$IC(M_k) = 2 \sum_{t=1}^T \log f_{k,t}(Y_t| \widehat{\theta_k}) - c_{T,k}$$
where $c_{T,k}$ is the penalty term for $M_k$. 
The question we will explore is how different choices of the penalty term $c_{T,k}$ give rise to criteria that behave in different ways. 



\subsection{Weak Consistency}
To begin, suppose that, among the candidate models, there is a \emph{unique} KL-minimizing model among the candidates.
We say that a model selection criterion is \textbf{weakly consistent} if it selects this KL-minimizing candidate model with probability approaching one as $T\rightarrow \infty$.

\paragraph{Sufficient Conditions for Weak Consistency}
Suppose that exactly one of the candidates minimizes the KL distance: call it $M_{k_0}$. 
To state this precisely, suppose that
	$$\underset{T\rightarrow \infty}{\lim\inf}\left(\underset{k \neq k_0}{\min} \frac{1}{T}\sum_{t = 1}^T \left\{ KL(g; f_{k,t}) - KL(g;f_{k_0,t}) \right\} \right) > 0$$
Then, if $c_{T,k}> 0$ and $c_{T,k} = o_p(T)$, $IC(M_k)$ is \emph{weakly consistent}: it selects $M_{k_0}$ with probability approaching one in the limit.
Weak consistency continues to hold if the penalty term $c_{T,k}$ equals zero for one of the models, so long as it is strictly positive for all of the others.

\paragraph{AIC and BIC are Weakly Consistent}
since both satisfy $T^{-1}c_{T,k} \overset{p}{\rightarrow} 0$.
	\begin{eqnarray*}
		\mbox{BIC Penalty:}&& c_{T,k} = \log(T) \times \mbox{length}(\theta_k)\\
		\mbox{AIC Penalty:} && c_{T,k} = 2\times \mbox{length}(\theta_k)
	\end{eqnarray*}

\subsection{Consistency}
But what if \emph{two or more} models minimize the KL-divergence? We very often use information criteria to select among \emph{nested models} to decide, for example, whether to restrict certain elements of $\theta$ to be equal to zero. 
Suppose we want to choose the number of lags to include in an AR model. 
The usual way to do this is to specify a maximum lag-length, say 3 periods, and then evaluate each of the AR models up to this order: AR(1), AR(2), and AR(3). 
But in this example is is entirely possible that the KL minimizer will \emph{fail} to be unique. 
The AR(2) model is just a special case of the AR(3) with one coefficient set equal to zero. 
Similarly, the AR(1) model is just a special case of the AR(2). 
Stated mode generally, if an AR(k) model with all coefficients different from zero is the KL minimizer, then an AR(k+1) model also minimizes the KL divergence, as does an AR(k+2) and an AR(k+3) by setting certain coefficients to zero. 
In situations like this, where there is a tie in the KL divergence, it makes sense to choose the most ``parsimonious'' specification, in other words the one with the fewest parameters. 
This idea is often called \textbf{consistency}.


\paragraph{Sufficient Conditions for Consistency}
Suppose that, among our set of candidate models there is a tie in the KL divergence. Let $\mathcal{J}$ be the set of all models that attain the minimum KL divergence. Among these, let $\mathcal{J}_0$ denote the subset with the minimum number of parameters. \emph{Either} of the following two conditions is sufficient for consistency. In other words, both (a) and (b) imply that we will select a model from $\mathcal{J}_0$ with probability approaching one in the limit:
	$$\underset{T\rightarrow \infty}P\left\{ \underset{\ell \in \mathcal{J}\backslash \mathcal{J}_0}{\min} \left[ IC(M_{j_0}) - IC(M_\ell)\right] > 0 \right\} = 1$$
Here are the alternative sets of conditions: 
	\begin{enumerate}[(a)]
		\item The following two conditions are sufficient for consistency:
			\begin{enumerate}[(i)]
				\item For all $k \neq \ell \in \mathcal{J}$ 
					$$\underset{T\rightarrow \infty}{\lim\sup} \frac{1}{\sqrt{T}} \sum_{t=1}^T\left\{ KL(g; f_{k,t}) - KL(g;f_{\ell,t}) \right\}<\infty$$
				\item For all $j_0 \in \mathcal{J}_0$ and $\ell \in (\mathcal J \backslash \mathcal{J}_0)$
					$$P\left\{\left(c_{T,\ell} - c_{T,j_0} \right)/\sqrt{T} \rightarrow \infty \right\}= 1$$
			\end{enumerate}
		\item The following two conditions are \emph{also} sufficient for consistency:
			\begin{enumerate}[(i)]
				\item For all $k \neq \ell \in \mathcal{J}$ 
					$$\sum_{t=1}^T \left[\log f_{k,t}(Y_t|\theta^*_k) - \log f_{\ell,t}(Y_t|\theta^*_\ell) \right] = O_p(1)$$
				where $\theta^*_k$ and $\theta^*_\ell$ are the respective KL minimizing parameter values.
				\item For all $j_0 \in \mathcal{J}_0$ and $\ell \in (\mathcal J \backslash \mathcal{J}_0)$
					$$P\left(c_{T,\ell} - c_{T,j_0} \rightarrow \infty \right) = 1$$
			\end{enumerate}
	\end{enumerate}
Note that each of these alternative sets of conditions has \emph{two parts}: the first is a regularity condition that restricts the asymptotic behavior of the models in $\mathcal{J}$ while the second is a condition on the penalty term $c_{T,k}$. We immediately see that the penalty terms for the AIC and TIC \emph{cannot} satisfy (a)(ii) or (b)(ii) since $(c_{T,\ell} - c_{T,j_0})$ \emph{does not depend on sample size}. While this does not consitute a proof, it does turn out that neither is consistent: even in the limit AIC and TIC have a non-zero probability of ``overfitting,'' i.e.\ selection a model that is in $\mathcal{J}\backslash \mathcal{J}_0$. In constrast, under (b)(i) the BIC \emph{is consistent} since
	$$c_{T,\ell} - c_{T,j_0} = \log(T) \left\{\mbox{length}(\theta_\ell) - \mbox{length}(\theta_{j_0}) \right\}$$
The term in braces is \emph{positive} since $\ell \in\mathcal{J}\backslash \mathcal{J}_0$, i.e.\ $\ell$ is not as parsimonious as $j_0$, and $\log(T) \rightarrow \infty$. This means that in the limit, BIC will \emph{always} select a model in $\mathcal{J}_0$.

\paragraph{What about selecting the true DGP?}
The way we will just defined consistency did \emph{not} in fact require that the true DGP is among the models under consideration. If the true DGP \emph{is} among the models in our set, however, the preceding result gives conditions under which we are guaranteed to select it in the limit. Why is this the case? First of all, the true DGP minimizes the KL and the minimized value is zero. (See the notes for Lecture 1.) The only way that \emph{another} model could also minimize the KL divergence in this case is if it has ``superfluous'' parameters. For example, suppose the true DGP is an AR(1) but we also consider an AR(2). Hence, the true DGP is necessarily the most parsimonious model among those that minimize the KL divergence. 


\section{Efficient Model Selection}
Roughly speaking, a model selection criterion is called efficient if it performs ``nearly as well'' as the theoretical optimum relative to some loss function. To make this more concrete, we'll look at a particular example.

Let $\left\{\epsilon_t\right\}_{-\infty}^{\infty}$ by an iid sequence of $N(0,\sigma^2)$ random variables and let $\{X_t\}$ be a stationary Gaussian Process that satisfies
$$X_t + a_1 X_{t-1} + a_2 X_{t-2} + \cdots  = \epsilon_t$$  
for some set of coefficients $\{a_j\}$. We attempt to approximate this stochastic process with an AR$(k)$ model, namely
$$X_t + a_1 X_{t-1} + \cdots + a_2 X_{t-k}  = \epsilon_t$$  
and calculate estimates $\widehat{a}_1, \hdots, \widehat{a}_k$ using observations $X_1, \hdots, X_T$. Now, suppose our goal is to make good one-step-ahead forecasts where ``good'' means minimum mean-squared prediction error. To keep things simple it is typically assumed that we have a \emph{new} realization $Y_1, \hdots, Y_T$ of the \emph{same} time series that is independent of $X_1, \hdots, X_T$. This is indeed an unrealistic assumption, but it simplifies various calculations. Although it's possible to proceed without it, you'll often see it invoked. The one-step-ahead prediction is
	$$\widehat{Y}_{t+1} = -\widehat{a}_1 Y_t - \widehat{a}_2 Y_{t-2} - \cdots - \widehat{a}_k Y_{t-k+1}$$
as the one-step-ahead mean-squared prediction error is
	\begin{eqnarray*}
		MSPE(k) &=& E\left[ \left(Y_{t+1} - \widehat{Y}(k)_{t+1}\right)^2\left. \right| X_1, \hdots, X_T\right]
	\end{eqnarray*}
Our ideal would be to estimate an $AR(k^*)$ model for forecasting where $k^*$ minimizes $MSPE(k)$. Since we don't know $k^*$ we use a model selection criterion to estimate it. Let $\widehat{k}$ be the lag-length that is selected by some model selection criterion. We say that this criterion is \emph{asymptotically efficient} if
	$$\frac{MSPE(\widehat{k})}{MSPE(k^*)}\overset{p}{\rightarrow} 1 \quad \mbox{as} \quad T \rightarrow \infty$$
Under appropriate assumptions, it can be shown for this example that the AIC and AIC$_c$ are asymptotically efficient while the BIC is not.
To get this result to work, we need an asymptotic framework in which none of the candidate models provides ``too good'' of an approximation to the true DGP.
The way to get this to work is to suppose that the true AR order is \emph{growing with sample size}.
For more discussion, see Leeb and P\"{o}tscher (2009a).



\section{A Simple Example}
Let $Y_1, \hdots, Y_T \overset{\mbox{iid}}{\sim} N(\mu,1)$ and consider two models: $M_0$ assumes that $\mu = 0$ while $M_1$ doesn't make any assumption about the value of $\mu$. Now suppose we want to use an information criterion to choose between $M_0$ and $M_1$. We'll consider penalty terms of the form $c_{T,k} = d_T 
\times \mbox{length}(\theta_k)$ which includes both the AIC and BIC as special cases. Since $M_0$ has \emph{zero} parameters while $M_1$ has one parameter, our information criteria are as follows:
	\begin{eqnarray*}
		IC_0 &=& 2 \max_\mu \left\{\ell_T(\mu)\colon M_0 \right\} \\
		IC_1 &=& 2 \max_\mu \left\{\ell_T(\mu)\colon M_1 \right\} - d_T
	\end{eqnarray*}

\begin{eqnarray*}
	\ell_T(\mu)&=& \sum_{t=1}^T \log \left( \frac{1}{2\pi} \exp \left\{-\frac{1}{2}(Y_t - \mu)^2 \right\}\right)\\
	&\vdots& \boxed{\mbox{fill in later}}\\
	&=& -\frac{T}{2} \left\{ \widehat{\sigma}^2 + \log(2\pi)\right\} - \frac{T}{2}\left(\bar{Y} - \mu \right)^2\\
	&=& C - \frac{T}{2}\left(\bar{Y} - \mu \right)^2
\end{eqnarray*}
Hence, substituting $0$ for $\mu$ under $M_0$ and the MLE $\bar{Y}$ for $\mu$ under $M_1$, we have
	\begin{eqnarray*}
		IC_0 &=& 2 \max_\mu \left\{\ell_T(\mu)\colon M_0 \right\} = 2C - T\bar{Y}^2\\
		IC_1 &=&2 \max_\mu \left\{\ell_T(\mu)\colon M_1 \right\} - d_T = 2C - d_T
	\end{eqnarray*}
Therefore,
	$$IC_1 - IC_0 = T\bar{Y}^2 - d_T$$
and we choose $M_1$ if this quantity is positive, in other words if
	\begin{eqnarray*}
		T\bar{Y}^2 &\geq& d_T\\
		\left|\sqrt{T} \bar{Y} \right| &\geq& \sqrt{d_T}
	\end{eqnarray*}
Thus, our selected model is 
	$$\widehat{M} = \left\{\begin{array}
		{cc} M_1, & |\sqrt{T}\bar{Y} | \geq \sqrt{d_T} \\
		M_0, & |\sqrt{T}\bar{Y} |  < \sqrt{d_T}
	\end{array} \right.$$
and our \emph{post-selection estimator} is 
	$$\widehat{\mu}=\left\{\begin{array}
		{cc} \bar{Y}, & |\sqrt{T}\bar{Y} | \geq \sqrt{d_T} \\
		0, & |\sqrt{T}\bar{Y} | < \sqrt{d_T}
		\end{array}\right.$$
For the AIC we have $d_T = 2$ while for the BIC we have $d_T = \log(T)$. Now let's examine the asymptotics as $T \rightarrow \infty$.

\paragraph{Case I: $\mu \neq 0$} In this case, $M_1$ is the true DGP and the unique KL-minimizer. Since it's the true DGP, the KL divergence for $M_1$ is exactly zero. (See Lecture 1.) We can calculate the KL divergence for $M_0$ using similar steps to those we employed to derive the AIC$_c$ in Lecture 2. First, we have
	\begin{eqnarray*}
		\log g(y) - \log f_\theta(y) &=& -\frac{1}{2}(y-\mu)^2 + \frac{1}{2} y^2\\
		&\vdots& \boxed{\mbox{a little algebra}}\\
		&=& \mu \left(y - \frac{\mu}{2}\right)
	\end{eqnarray*}
since the constant $-\log(2\pi)/2$ appears in each term and hence cancels. Thus,
	\begin{eqnarray*}
		KL(g;f_\theta) &=& \int \frac{1}{\sqrt{2\pi}} \exp\left\{ -\frac{1}{2} (y-\mu)^2\right\} \mu\left(y - \frac{\mu}{2} \right)\; dy\\
			&\vdots& \boxed{\mbox{fill in later}}\\
			&=& \frac{\mu^2}{2}
	\end{eqnarray*}
To summarize, we have
	\begin{eqnarray*}
		KL(g;M_1) &=&0\\
		KL(g;M_0) &=& \frac{\mu^2}{2}
	\end{eqnarray*}
Now let's check our sufficient conditions for weak consistency. First, we have
	\begin{eqnarray*}
		\underset{T\rightarrow \infty}{\lim\inf}\left(\underset{k \neq k_0}{\min} \frac{1}{T}\sum_{t = 1}^T \left\{ KL(g; f_{k,t}) - KL(g;f_{k_0,t}) \right\} \right) &=& \underset{n\rightarrow \infty}{\lim\inf}\ \frac{1}{T}\sum_{t = 1}^T  \left(\frac{\mu^2}{2} - 0\right) \\
		&=&  \underset{T\rightarrow \infty}{\lim\inf}\left(\frac{\mu^2}{2}\right) >0
	\end{eqnarray*}
as required. Now, the condition on the penalty term is $c_{T,k} = o_p(T)$, in other words $c_{T,k}/T \overset{p}{\rightarrow} 0$ both the AIC and BIC penalties satisfy this condition. Hence, if $M_1$ is the true model, both the AIC and BIC will select it with probability approaching 1 as $T\rightarrow \infty$.

\paragraph{Case II: $\mu = 0$} In this case, both $M_1$ and $M_0$ are true and \emph{both} minimize the KL divergence. The most parsimonious model, however, is $M_0$. Hence, using our notion of consistency (\emph{not} weak consistency), we'd like our criteria to select $M_0$. We'll use the second set of sufficient conditions for consistency. In this example, it's easy to verify (b)(i). Since a $N(0,1)$ model is nested inside a $N(\mu,1)$ model, if the true distribution is $N(0,1)$ then the likelihood ratio statistic is asymptotically $\chi^2(1)$, hence the log-likelihood ratio is $O_p(1)$ as required. 

We know from above that the AIC penalty does \emph{not} satisfy (b)(ii) but the BIC penalty \emph{does}. Hence the BIC will select $M_0$ with probability approaching one in the limit.

\paragraph{Finite Sample Selection Probabilities} Since this is such a simple example, we can do better than appeal to asymptotics: we can calculate the exact finite-sample behavior of the selection criteria. The AIC penalty is $2 \times \mbox{length}(\theta)$ which corresponds to $d_T = 2$. Hence, the AIC-selected model is
	$$\widehat{M}_{AIC} = \left\{\begin{array}
		{cc} M_1, &|\sqrt{T}\bar{Y}| \geq \sqrt{2} \\
		M_0, & |\sqrt{T} \bar{Y}| < \sqrt{2}
	\end{array} \right.$$
Hence,
	\begin{eqnarray*}
		P\left(\widehat{M}_{AIC} = M_1\right) &=& P\left(\left|\sqrt{T}\bar{Y} \right| \geq \sqrt{2}  \right)\\
		&=& P\left(\left|\sqrt{T}\mu + Z\right| \geq \sqrt{2}  \right)\\
		&=& P\left(\sqrt{T}\mu + Z \leq -\sqrt{2}\right) + \left[1 - P\left(\sqrt{T} \mu +Z \leq \sqrt{2}\right) \right]\\
			&=& \Phi\left(-\sqrt{2} - \sqrt{T}\mu\right) + \left[1 -  \Phi\left(\sqrt{2} - \sqrt{T} \mu \right)\right]
	\end{eqnarray*}
where $Z \sim N(0,1)$ using the fact that $\bar{Y} \sim N(\mu, 1/T)$ since $Var(Y_t)=1$.

Now, the BIC penalty is $\log(T)\times \mbox{length}(\theta)$ which corresponds to $d_T = \log(T)$. Hence, the BIC-selected model is
	$$\widehat{M}_{BIC} = \left\{\begin{array}
		{cc} M_1, & |\sqrt{T}\bar{Y} | \geq \sqrt{\log(T)} \\
		M_0, & |\sqrt{T} \bar{Y}| < \sqrt{\log(T)}
	\end{array} \right.$$
Using the exact same steps as for the AIC except with $\sqrt{\log(T)}$ in the place of $\sqrt{2}$, we have
	\begin{eqnarray*}
		P\left(\widehat{M}_{BIC} = M_1\right) &=& P\left(\left|\sqrt{T}\bar{Y} \right| \geq \sqrt{\log(T)}  \right)\\
			&=& \Phi\left(-\sqrt{\log(T)} - \sqrt{T}\mu\right) + \left[1 -  \Phi\left(\sqrt{\log(T)} - \sqrt{T} \mu \right)\right]
	\end{eqnarray*}

\todo[inline]{Shiny App: \url{http://glimmer.rstudio.com/fditraglia/CH_Figure_4_1/}}

\paragraph{What is the probability of overfitting?} Suppose $\mu = 0$. In this case both models are KL-minimizers but we'd prefer $M_0$ since it's more parsimonious. For a generic information criterion of the form we're considering here, we calculate the ``probability of overfitting'' as follows
\begin{eqnarray*}
	P\left(\widehat{M} = M_1\right) &=& P\left(|\sqrt{T}\bar{Y}|\geq \sqrt{d_T}\right) = P(|Z|\geq \sqrt{d_T})\\
	 &=& P(Z^2 \geq d_T) = P(\chi^2_1 \geq d_T)
\end{eqnarray*}
where $Z \sim N(0,1)$. For the AIC $d_T = 2$ so the probability of overfitting is $P(\chi^2_1 \geq 2)\approx 0.157$. For the BIC $d_T = \log(T)$ so the probability of overfitting is $P(\chi^2_1 \geq \log T) \rightarrow 0$ as $T\rightarrow 0$.

\paragraph{The Post-Selection Estimator}

	$$\widehat{\mu}=\left\{\begin{array}
		{cc} \bar{Y}, & |\sqrt{T}\bar{Y} | \geq \sqrt{d_T} \\
		0, & |\sqrt{T}\bar{Y} | < \sqrt{d_T}
		\end{array}\right.$$
Consider MSE risk, scaling up by $T$ since variances for well-behaved problems are $O(1/T)$. Recall from above that $\sqrt{T} \bar{Y} = \sqrt{T}\mu +Z$ where $Z\sim N(0,1)$. Thus,
\begin{eqnarray*}
	R_T(\mu) &=& T E_\mu\left[\left( \widehat{\mu} - \mu\right)\right] = E_\mu\left[\left(\sqrt{T} \widehat{\mu} - \sqrt{T} \mu\right)\right]\\
		&=& E\left\{\left[ \left(\sqrt{T}\mu + Z \right)\mathbf{1}\left\{\sqrt{T}\mu + Z \geq \sqrt{d_T}\right\}- \sqrt{T}\mu \right]^2 \right\}\\
		&\vdots& \boxed{\mbox{fill in later}}\\
		&=& 1 - \int_{a}^{b} z^2 \phi(z)\; dz + T\mu^2 \left[\Phi(b) - \Phi(a) \right]
\end{eqnarray*}
where
	\begin{eqnarray*}
		a &=& -\sqrt{d_T} - \sqrt{T}\mu\\
		b &=& \sqrt{d_T} - \sqrt{T}\mu
	\end{eqnarray*}
To evaulate this risk function, we need an explicit formula for the integral that makes up the second term. This sounds like a job for integration by parts! We'll take $u = -z$ and $dv = -z \exp\{-z^2/2\}$ since
	$$\frac{d}{dz} \left(\exp\left\{-\frac{z^2}{2}\right\}\right) = -z\exp\left\{-\frac{z^2}{2}\right\}$$
Thus, $v = \exp\{-z^2/2\}$, $du = -1$ and we have
\begin{eqnarray*}
	\int_a^b z^2 \phi(z) \; dz &=&\frac{1}{\sqrt{2\pi}} \int_a^b z^2 \exp\left\{-\frac{z^2}{2} \; dz\right\} \\
		&=& \frac{1}{\sqrt{2\pi}}\left[-z \exp\left\{ \left.-\frac{z^2}{2}\right\}\right|_a^b+ \int_a^b \exp\left\{-\frac{z^2}{2}\right\}  \; dz \right]\\
		&=& a\phi(a) - b\phi(b) + \Phi(b) - \Phi(a)
\end{eqnarray*}
Putting it all together, we have
	\begin{eqnarray*}
		R_T(\mu) &=& 1 - \left[a\phi(a) - b\phi(b) + \Phi(b) - \Phi(a) \right] + T\mu^2 \left[\Phi(b) - \Phi(a) \right]\\
		&=&1 + \left[b\phi(b) - a\phi(a)\right]  + (T\mu^2 - 1) \left[\Phi(b) - \Phi(a) \right] 
	\end{eqnarray*}
where
	\begin{eqnarray*}
		a &=& -\sqrt{d_T} - \sqrt{T}\mu\\
		b &=& \sqrt{d_T} - \sqrt{T}\mu
	\end{eqnarray*}
\todo[inline]{Shiny App: \url{http://glimmer.rstudio.com/fditraglia/CH_Figure_4_2/}}

What we see from the picture is that, in exchange for low risk in a small neighborhood of $\mu=0$, BIC has max risk that \emph{diverges} as the sample size increases. In contrast, AIC has bounded max risk. This isn't just a problem with BIC: \emph{any} consistent selection criterion will suffer from this defect.

\paragraph{Postscript} The preceding is a special example of a more general phenomenon: consistency and efficiency are mutually exclusive properties. In general, consistent model selection criterion will have unbounded minimax risk. There is a huge literature on this topic but it's fairly technical. The key observation is that pointwise and uniform risk approximations give very different results. Yang (2007) gives a readable introduction. See Leeb and P\"{o}tscher (2009) for a comprehensive list of references. 



 
