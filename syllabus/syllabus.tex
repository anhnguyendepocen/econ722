\documentclass[11pt, letterpaper]{article}
\usepackage{geometry}
\geometry{margin=1in} 
\usepackage{setspace}
\linespread{1}
\usepackage{hyperref}
\usepackage{totcount}
\usepackage{termcal}
\usepackage{enumerate}
\usepackage{fancybox}
\usepackage{amsmath, amssymb}

%%%%%%%%%%%%% A Command to create an automatically numbered Quiz Icon in the course calendar.
\newtotcounter{quizcounter}
\newcommand{\quiz}{\addtocounter{quizcounter}{1}\fbox{\textbf{Quiz \#\arabic{quizcounter}}}}


% Some useful commands (our classes always meet either on Monday and Wednesday 
% or on Tuesday and Thursday)
\newcommand{\MWClass}{%
\calday[Monday]{\classday} % Monday
\skipday % Tuesday (no class)
\calday[Wednesday]{\classday} % Wednesday
\skipday % Thursday (no class)
\skipday % Friday 
\skipday\skipday % weekend (no class)
}

\newcommand{\TRClass}{%
\skipday % Monday (no class)
\calday[Tuesday]{\classday} % Tuesday
\skipday % Wednesday (no class)
\calday[Thursday]{\classday} % Thursday
\skipday % Friday 
\skipday\skipday % weekend (no class)
}

\newcommand{\Holiday}[2]{%
\options{#1}{\noclassday}
\caltext{#1}{#2}
}

\begin{document}


\thispagestyle{plain}

\begin{center}
\Large
\sc
Econometrics IV\\
\large
Econ 722\\
\large
Spring 2018
\end{center}



\normalsize

\noindent \textbf{Course Instructor:} Francis DiTraglia \\
Office: MCNB 535\\
Office Hours: M 2--4pm 

\medskip

 
\noindent \textbf{Lecture Time and Location:} 
TR 1:30--3PM MCNB 169 

\medskip
 
\noindent \textbf{Course Website:} \url{http://ditraglia.com/econ722} 

\medskip



\noindent \textbf{Course Description:} 
Talk about how this is a little different from the regular Econ 722 since it's only a half course.
Still relevant for macroeconometrics and time series but will also consider some cross-section applications.
Broaden appeal.
Applied microeconomics students who are interested in understanding some of the techniques related to ``big data'' will find the course helpful.



\medskip

\noindent \textbf{Prerequisites:} 
The prerequisites for this course are Econ 705 and 706 or equivalent graduate level econometrics. 



\medskip

\noindent \textbf{Recommended Text:} 
Mostly articles and other references that I will post online.
Comprehensive reference for much of the machine learning material is
Murphy, Kevin (2012): {\em Machine Learning: A Probabilistic Perspective}

\medskip


\section*{Assignments and Grading}

Grades for this course will be determined based on\dots 
	\begin{equation*}
	\begin{split}
		\mbox{Overall Score} = (30\% \times \mbox{Quizzes})  + (20\% \times \mbox{Midterm 1}) + (20\% \times \mbox{Midterm 2}) + (30\% \times \mbox{Final}).
	\end{split}
	\end{equation*}
You can earn extra credit worth up to 5\% of your course grade: see ``Extra Credit'' for details.

\medskip 


\noindent \textbf{Econometrics Workshop:} 




%NOTE: don't use leading zeros in dates! In other words, use 1/1/2014 rather than 01/01/2014
 
\newpage

\begin{center}
\small
\begin{calendar}{3/12/2018}{7} %Date of Monday in first week of classes, NOT the date of the first class!
\setlength{\calboxdepth}{.25in}
\TRClass
% schedule
\caltexton{1}{Computing for Econometrics: Creating R Packages, Rcpp, Parallel R}
\caltextnext{Model Selection I: AIC, TIC, Corrected AIC}
\caltextnext{Model Selection II: Mallow's $C_p$, Cross-validation, Examples} 
\caltextnext{Model Selection III: Focused Info.\ Criterion, Asymptotic Properties}
\caltextnext{Moment Selection I: Consistent Moment Selection, Andrews (1999) etc.}
\caltextnext{Moment Selection II: Focused Moment Selection and Averaging, Inference} 
\caltextnext{High-Dimensional Regression I: James-Stein, Ridge Regression, PCR}
\caltextnext{High-Dimensional Regression II: LASSO}
\caltextnext{Factor Models I: Factor Analysis vs.\ PCA, High-dimensional Factor Models} 
\caltextnext{Factor Models II: Diffusion Index Forecasting, Factor Choice, Factors as IVs}
\caltextnext{Additional Topics I (Time Permitting)}
\caltextnext{Additional Topics II (Time Permitting)}
\caltextnext{Additional Topics III (Time Permitting)}
% ... and so on

% Holidays
%\Holiday{1/9/2018}{\textbf{Winter Break -- No Lecture}}

%\options{2/13/2018}{\noclassday} % first midterm exam
%\caltext{2/13/2018}{\shadowbox{\textbf{Midterm I} -- Material through Feb.\ 8th}}
\options{4/26/2018}{\noclassday} % finals
\caltext{4/26/2018}{\textbf{Reading Day -- No Lecture}}
\end{calendar}
\end{center}


\end{document}
