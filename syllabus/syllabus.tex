\documentclass[11pt, letterpaper]{article}
\usepackage{geometry}
\geometry{margin=1in} 
\usepackage{setspace}
\linespread{1}
\usepackage{hyperref}
\usepackage{totcount}
\usepackage{termcal}
\usepackage{enumerate}
\usepackage{fancybox}
\usepackage{amsmath, amssymb}

%%%%%%%%%%%%% A Command to create an automatically numbered Quiz Icon in the course calendar.
\newtotcounter{quizcounter}
\newcommand{\quiz}{\addtocounter{quizcounter}{1}\fbox{\textbf{Quiz \#\arabic{quizcounter}}}}


% Some useful commands (our classes always meet either on Monday and Wednesday 
% or on Tuesday and Thursday)
\newcommand{\MWClass}{%
\calday[Monday]{\classday} % Monday
\skipday % Tuesday (no class)
\calday[Wednesday]{\classday} % Wednesday
\skipday % Thursday (no class)
\skipday % Friday 
\skipday\skipday % weekend (no class)
}

\newcommand{\TRClass}{%
\skipday % Monday (no class)
\calday[Tuesday]{\classday} % Tuesday
\skipday % Wednesday (no class)
\calday[Thursday]{\classday} % Thursday
\skipday % Friday 
\skipday\skipday % weekend (no class)
}

\newcommand{\Holiday}[2]{%
\options{#1}{\noclassday}
\caltext{#1}{#2}
}

\begin{document}


\thispagestyle{plain}

\begin{center}
\Large
\sc
Econometrics IV\\
\large
Econ 722\\
\large
Spring 2018
\end{center}



\normalsize

\noindent \textbf{Course Instructor:} Francis DiTraglia \\
Office: MCNB 535\\
Office Hours: M 2--4pm 

\medskip

 
\noindent \textbf{Lecture Time and Location:} 
TR 1:30--3PM MCNB 169 

\medskip
 
\noindent \textbf{Website and Courseware:} Lecture slides and notes, as well as problem sets and a copy of this syllabus will be posted at \url{http://ditraglia.com/econ722}.
Course announcements will be posted on Canvas: \url{http://canvas.upenn.edu}.
Please submit your problem sets electronically via Canvas rather than handing in paper copies.
Course readings will be shared via Dropbox.

\medskip



\noindent \textbf{Course Description:} 
In recent years Econ 722 has been divided into two parts.
The first part, taught by Frank Schorheide, covered Bayesian analysis of vectorautoregressions (VARs) while the second part, taught by me, covered model selection as well as selected methods from ``big-data'' econometrics and machine learning. 
In the past it has not been possible to register for only one half of Econ 722.
As such, the audience for the course has consisted mainly of students who intend to pursue research in either econometrics or empirical macroeconomics. 
Because Frank Schorfheide is on leave for the 2017--2018 academic year, however, I will teach what had historically been the second half of Econ 722 as a standalone course in the Spring of 2018.
Although Econ 722 is listed as ``Advanced Topics in Time Series Econometrics'' in the course catalogue, most of the material from my half of the course is applicable to both time-series and cross-section settings. 
Given this semester's unusual arrangement, I would like to encourage empirical microeconomics students interested in ``big data'' and machine learning to consider registering for Econ 722 in the Spring of 2018.
While the course will remain relevant for students interested in macroeconometrics, I plan to include a broader range of examples this semester.
Graded assignments will also include an option for those whose interests lie outside of time series to apply the course material to a setting more closely related to their research.
The semester calendar on the last page of this document lists the topics I covered when teaching the second half of Econ 722 in Spring 2017.
While subject to change, this gives a good overview of the core course material.
Time permitting, I will allocate three lectures to ``Additional Topics,'' tailored to the research interests of the students who register for the course.
Possible topics include high-performance computing, post-selection inference, model averaging, and additional methods from machine learning with economic applications.

\medskip

\noindent \textbf{Prerequisites:} 
Econ 705 and 706 or equivalent graduate level econometrics. 



\medskip

\noindent \textbf{Readings:} 
The readings for this course will include my lecture slides and notes, as well as research papers that I will share with you via a password protected Dropbox folder.
The recommended text for this course is \emph{Machine Learning: A Probabilistic Perspective} by Kevin Murphy (2012).

\medskip

\section*{Assignments and Grading}
	\begin{equation*}
	\begin{split}
    \mbox{Grade} = (25\% \times \mbox{Referee Report}) + (25\% \times \mbox{Presentation})  + (50\% \times \mbox{Problem Sets \& Participation}) 
	\end{split}
	\end{equation*}


\medskip

\noindent \textbf{Referee Report:} 
Each student will be assigned a recent working paper for which to write a referee report between 3--5 pages in length.
The report should provide a detailed summary of the paper, along with specific comments on how it could be improved.
Details will be discussed in class.

\medskip 
\noindent \textbf{Presentation:} 
Each student will give an in-class presentation, with slides, of approximately 20 minutes in length.
Presentations will take place in the second half of April.
The topic of the presentation must be related to the course material and agreed upon with the instructor in advance. 
Your presentation may take the form of either (1) a literature review and proposal for a research paper or (2) a summary of a paper or method related to but not covered in Econ 722.
If you elect to present a paper, it must be different from the one you were assigned for your referee report. 
As you will be graded on both the quality of your content and your presentation skills, I will spend some time in class discussing how to give a good talk and provide you with relevant readings. 

\medskip 
\noindent \textbf{Problem Sets and Class Participation:} 
I will assign four problem sets during the semester.
You are also expected to study the assigned readings, attend class, and participate in discussions.





%NOTE: don't use leading zeros in dates! In other words, use 1/1/2014 rather than 01/01/2014

\section*{Semester Calendar}

\begin{center}
\small
\begin{calendar}{3/12/2018}{7} %Date of Monday in first week of classes, NOT the date of the first class!
\setlength{\calboxdepth}{.25in}
\TRClass
% schedule
%\caltexton{1}{Computing for Econometrics: Creating R Packages, Rcpp, Parallel R}
\caltexton{1}{Model Selection I: AIC, TIC, Corrected AIC}
\caltextnext{Model Selection II: Mallow's $C_p$, Bayesian Model Comparison, BIC}
\caltextnext{Model Selection III: Cross-validation}
\caltextnext{Model Selection IV: Asymptotic Properties, Consistent vs.\ Efficient Model Selection}
\caltextnext{Moment Selection I: Consistent Moment Selection, Andrews (1999) etc.}
\caltextnext{Moment Selection II: Focused Moment Selection and Averaging, Inference} 
\caltextnext{High-Dimensional Regression I: The Stein Phenomenon, QR and SV Decompositions, PCA}
\caltextnext{Ridge Regression, PCR, LASSO}
\caltextnext{Factor Models: Diffusion Index Forecasting, Factor Choice, Factors as IVs}
\caltextnext{Additional Topics I (Time Permitting)}
\caltextnext{Additional Topics II (Time Permitting)}
\caltextnext{Additional Topics III (Time Permitting)}
\caltextnext{Student Presentations}
% ... and so on

% Holidays
%\Holiday{1/9/2018}{\textbf{Winter Break -- No Lecture}}

%\options{2/13/2018}{\noclassday} % first midterm exam
%\caltext{2/13/2018}{\shadowbox{\textbf{Midterm I} -- Material through Feb.\ 8th}}
\options{4/26/2018}{\noclassday} % finals
\caltext{4/26/2018}{\textbf{Reading Day -- No Lecture}}
\end{calendar}
\end{center}


\end{document}
