\documentclass[12pt]{article}
\usepackage[margin=1.5in]{geometry}
\usepackage{todonotes}
\usepackage{amssymb,amsmath,amsthm,graphicx}
\usepackage{rotating}
\usepackage{setspace}
\usepackage{enumerate}
\usepackage{graphicx}
\usepackage{listings}
\usepackage{multirow}
\usepackage{color}
\usepackage{threeparttable}
\usepackage{caption}
\usepackage{subcaption}

\newtheorem{assump}{Assumption}[section]
\newtheorem{pro}{Proposition}[section]
\newtheorem{lem}{Lemma}[section]
\newtheorem{thm}{Theorem}[section]
\newtheorem{cor}{Corollary}[section]
\newtheorem{ineq}{Inequality}[section]
\newtheorem{defn}{Definition}[section]
\newtheorem{rem}{Remark}[section]
\newtheorem{ex}{Example}[section]
\theoremstyle{definition}
\newtheorem{alg}{Algorithm}[section]


\linespread{1.3}

\begin{document}

\title{Lecture 2: Various Model Selection Criteria}

\author{Francis J.\ DiTraglia}

\maketitle 


\section{The Corrected AIC}
To derive the TIC and AIC we used asymptotic theory to derive an analytical bias correction. These approximations tend to work well as long $n$ is fairly large relative to $p$ but when this is not the case, they can break down. We'll now consider an alternative that makes stronger assumptions and relies on \emph{exact} small-sample theory rather than asymptotics: the ``Corrected'' AIC, or AIC$
_c$, of Hurvich and Tsai (1989). Suppose that the true DGP is a linear regression model:
$$\textbf{y} = X\beta_0 + \boldsymbol{\epsilon}$$
where $\mathbf{\epsilon} \sim N(\mathbf{0}, \sigma_0^2 \mathbf{I}_T)$. Then $\mathbf{y}|X \sim N(X\beta_0, \sigma_0^2 \mathbf{I}_T)$ so the likelihood is
$$g(\textbf{y}|X;\beta_0, \sigma^2_0) = \left(2\pi\sigma_0^2\right)^{-T/2} \exp\left\{ -\frac{1}{2\sigma^2}(y - X\beta_0)'(y - X\beta_0)\right\}$$
and the log-likelihood is
$$\log\left[g(\textbf{y}|X;\beta_0, \sigma_0^2)\right] = -\frac{T}{2}\log(2\pi) -\frac{T}{2} \log(\sigma^2_0) - \frac{1}{2\sigma_0^2}\left(\textbf{y} - X\beta_0\right)'\left(\textbf{y} -X\beta_0\right)$$
Now suppose we evaluated the log-likelihood at some \emph{other} parameter values $\beta_1$ and $\sigma^2_1$. The vector $\beta_1$ might, for example, correspond to dropping some regressors from the model by setting their coefficients to zero, or perhaps adding in some additional regressors. We have
$$\log[f(\textbf{y}|X;\beta_1, \sigma_1^2)] = -\frac{T}{2}\log(2\pi) -\frac{T}{2} \log(\sigma^2_1) - \frac{1}{2\sigma_1^2}\left(\textbf{y} - X\beta_1\right)'\left(\textbf{y} -X\beta_1\right)$$
Since we've specified the density from which the data were generated as well as the density of the approximating model, we can \emph{directly calculate} the KL divergence rather than trying to find a reasonable large sample approximation. It turns out that for this example 
$$KL(g;f) = \frac{T}{2}\left[\frac{\sigma_0^2}{\sigma_1^2} - \log\left(\frac{\sigma_0^2}{\sigma_1^2}\right) - 1 \right] + \left(\frac{1}{2 \sigma_1^2}\right)\left(\beta_0 - \beta_1\right)'X'X\left(\beta_0 - \beta_1\right)$$
as you will demonstrate on the problem set. We need to estimate this quantity for it to be of any use in model selection. If let $\widehat{\beta}$ and $\widehat{\sigma}^2$ be the maximum likelihood estimators of $\beta_1$ and $\sigma_1^2$ and substitute them into the expression for the KL divergence, we have
\begin{eqnarray*}
\widehat{KL}(g;f) &=& \frac{T}{2}\left[\frac{\sigma_0^2}{\widehat{\sigma}^2} - \log\left(\frac{\sigma_0^2}{\widehat{\sigma}^2} \right) - 1\right] + \left(\frac{1}{2\widehat{\sigma}^2} \right)\left(\beta_0 - \widehat{\beta}\right)X'X\left(\beta_0 - \widehat{\beta}\right)
\end{eqnarray*}
We still have two problems. First, we haven't been entirely clear about what $\beta_1$ and $\sigma_1$ are. At the moment, they seem to be something like ``pseudo-true'' values. Second, and more importantly, we don't know $\beta_0$ and $\sigma_0^2$ so we can't use the preceding expression to compare models.

Hurvich and Tsai (1989) address both of these problems with the assumption that all models under consideration are \emph{at least correctly specified}. That is, while they may include a regressor whose coefficient is in fact zero, they do not exclude any regressors with a non-zero coefficient. This is the same assumption that we used above to reduce TIC to AIC. Under this assumption, $\beta_1$ and $\sigma_1^2$ \emph{are precisely the same} as $\beta_0$ and $\sigma_0^2$. More importantly, we can use all of the standard results for the exact finite sample distribution of regression estimators to help us. The idea is to construct an \emph{unbiased} estimator of the KL divergence. Taking expecations and rearranging slightly, we have
\begin{eqnarray*}
E\left[\widehat{KL}(g;f) \right] &=&\frac{T}{2}\left\{ E\left[\frac{\sigma_0^2}{\widehat{\sigma}^2} \right] - \log(\sigma_0^2) + E\left[\log(\widehat{\sigma}^2)\right] -1 \right\}\\
&& \quad + \; \frac{1}{2}E\left[\left(\frac{1}{\widehat{\sigma}^2} \right)\left(\widehat{\beta}-\beta_0 \right)X'X\left(\widehat{\beta}-\beta_0\right) \right]
\end{eqnarray*}
Now, under our assumptions $T\widehat{\sigma}^2/\sigma_0^2 \sim \chi^2_{T-k}$ where $k$ is the number of estimated coefficients in $\widehat{\beta}$. Further, if $Z \sim \chi^2_\nu$ then $E[1/Z] = 1/(\nu-2)$. It follows that
$$E\left[\frac{\sigma_0^2}{\widehat{\sigma}^2} \right] = E\left[\frac{T}{T\widehat{\sigma}^2/\sigma_0^2} \right] = \frac{T}{T - k - 2}$$
We can rewrite the final term similarly:
$$E\left[\left(\frac{1}{\widehat{\sigma}^2} \right)\left(\widehat{\beta}-\beta_0 \right)X'X\left(\widehat{\beta}-\beta_0\right) \right] = E\left[\left(\frac{T}{T\widehat{\sigma}^2/\sigma_0^2} \right)\frac{\left(\widehat{\beta}-\beta_0 \right)X'X\left(\widehat{\beta}-\beta_0\right)}{\sigma_0^2} \right]$$
Under our assumptions the two terms in the product are independent, so we can break apart the expectation. First, we have
$$E\left[\frac{T}{T\widehat{\sigma}^2/\sigma_0^2} \right] = \frac{T}{T - k - 2}$$
as above. For the second part,
$$\frac{\left(\widehat{\beta}-\beta_0 \right)X'X\left(\widehat{\beta}-\beta_0\right)}{\sigma_0^2} \sim \chi^2_k$$
and hence
$$E\left[\frac{\left(\widehat{\beta}-\beta_0 \right)X'X\left(\widehat{\beta}-\beta_0\right)}{\sigma_0^2} \right] = k$$
Putting all the pieces together,
\begin{eqnarray*}
E\left[\widehat{KL}(g;f) \right] &=&\frac{T}{2}\left\{ E\left[\frac{\sigma_0^2}{\widehat{\sigma}^2} \right] + \log(\sigma_0^2) - E\left[\log(\widehat{\sigma}^2)\right] -1 \right\}\\
&& \quad + \; \frac{1}{2}E\left[\left(\frac{1}{\widehat{\sigma}^2} \right)\left(\widehat{\beta}-\beta_0 \right)X'X\left(\widehat{\beta}-\beta_0\right) \right]\\
&=& \frac{T}{2} \left( \frac{T}{T-k-2} - \log(\sigma_0^2) + E\left[\log(\widehat{\sigma}^2)\right] -1\right) + \frac{T}{2}\left(\frac{k}{T - k -2}\right)\\
&=& \frac{T}{2} \left(\frac{T + k}{T - k -2} - \log(\sigma_0^2) + E\left[\log(\widehat{\sigma}^2)\right] -1\right)
\end{eqnarray*}
Since $\log(\widehat{\sigma}^2)$ is an unbiased estimator of $E[\log(\widehat{\sigma}^2)]$, substituting this give us an unbiased estimator of $E\left[\widehat{KL}(g;f) \right]$ as desired.
The only terms that vary across candidate models are the first and the third. Moreover, the multiplicative factor of $T/2$ does not affect model selection. Hence, the criterion is
$$AIC_c = \log(\widehat{\sigma}^2) + \frac{T + k}{T - k -2}$$
Note that the way this expression is written, \emph{smaller} values indicate a better model. So how does this compare to the plain-vanilla AIC for normal linear regression? The maximum likelihood estimators for this problem are
\begin{eqnarray*}
\widehat{\beta} &=& (X'X)^{-1}X'\mathbf{y}\\
\widehat{\sigma}^2 &=& \frac{(\mathbf{y} - X\widehat{\beta})'(\mathbf{y} - X\widehat{\beta})}{T}
\end{eqnarray*}
It follows that the maximized log-likehood is
\begin{eqnarray*}
\log\left[f(\mathbf{y}|X;\widehat{\beta})\right] &=& -\frac{T}{2} \log(\widehat{\sigma}^2) - \frac{1}{2\widehat{\sigma}^2}(y - X\widehat{\beta})'(y -X\widehat{\beta})\\
&=& -\frac{T}{2} \log(\widehat{\sigma}^2) - \frac{T}{2}
\end{eqnarray*}
by substituting $T\widehat{\sigma}^2$ for the numerator of the second term. Hence, the AIC for this problem is
$$AIC = 2\left(\ell(\widehat{\beta}) - k \right) = -T\log(\widehat{\sigma}^2) - T - 2k $$
But this way of writing things uses the \emph{opposite} sign convention from AIC$_c$. It's important to keep track of this, since different authors use different sign conventions for information criteria. To make the AIC comparable with our scaling of the AIC$_c$, we multiply through by $-1/T$ yielding
$$AIC = \log(\widehat{\sigma}^2) + \frac{T + 2k}{T}$$
where \emph{smaller} values now indicate a better model.


\section{Cross-Validation}
In the last lecture, we learned that choosing a model by minimizing the KL divergence is equivalent to choosing a model by maximizing the expected log likelihood. We also learned that the sample analogue
 $$E_{\widehat{G}}\left[\log f(\textbf{y}|\widehat{\theta}) \right]= \frac{\ell(\widehat{\theta})}{T}= \frac{1}{T} \sum_{t=1}^T \log f(y_t|\widehat{\theta})$$
provides a biased estimator of this quantity. Intuitively, the problem is that it uses the data twice: first to estimate $\widehat{\theta}$ and then to approximate the integral
	$$\int g(y) \log f(y|\widehat{\theta}) \; dy = E_G\left[\log f(Y_{new}|\widehat{\theta}) \right]$$
using the empirical CDF constructed from the sample observations. If the problem is that we've used the data twice, then an obvious idea is to find some way to use \emph{different} data for estimating $\theta$ than we use to approximate the integral. This is the idea behind cross-validation. We split the data into two parts, use one for estimation and the \emph{other} for model evaluation. To avoid ``wasting data'' we repeat this process sucessively for \emph{different} splits, so each observation has a chance to be used for for estimation and evaluation but \emph{never for both at the same time}.

To make this more concrete, we'll consider what is perhaps the most common form of cross-validation: ``leave-one-out'' cross-validation. For simplicity, suppose we have iid observations $Y_1, \hdots, Y_T$ and let $\widehat{\theta}_{(t)}$ denote the ML estimator for $\theta$ using all observations \emph{except} $Y_t$. The leave-one-out cross-validation estimator of the expected log likelihood is
	$$CV_T(1) = \frac{1}{T} \sum_{t=1}^T \log f(y_y|\widehat{\theta}_{(t)})$$
The idea is that, since the data are iid, $\widehat{\theta}_{(t)}$ is \emph{independent} of $Y_t$. Accordingly, the cross-validation estimate of the expected log-likelihood should \emph{not} be subject the the over-optimism problem that plagues the maximized log-likelihood. To use cross-validation for model selection, we simply calculate $CV(1)$ for the various models under consideration, and choose the one with the \emph{highest} value.

As it turns out, leave-one-out cross-validation is intimiately connected with the TIC. In fact the two are \emph{asymptotically equivalent}. To understand what's going on here, we'll need to use the properties of \emph{influence functions}, an idea from the literature on robust estiation. Let $H=H(G)$ be a functional and $G$ be some probability distribution. Then the influence function of $H$ at a point $y$ is defined as 
	$$\mbox{infl}(G,y) = \lim_{\epsilon \rightarrow 0} \frac{H\left(\left(1-\epsilon\right)G + \epsilon \delta\left(y\right)\right)}{\epsilon}$$
where $\delta(y)$ is the Dirac Delta Function, in this context a \emph{point mass} at $y$. This is a strange concept, so let's be more concrete. We're going to be considering the influence function for the following functional:
	$$\theta_0 = H(G) = \underset{\theta \in \Theta}{\arg \min} \;E_G\left[\log\left\{\frac{g(Y)}{f(Y|\theta)} \right\} \right]$$
What this says is that we can view $\theta_0$ as the result of applying an \emph{operator} $H$ to the distribution $G$. In this case $\theta_0$ is simply the pseudo-true value: the probability limit of the maximum likelihood estimator of $\theta$ based on $f(y|\theta)$. Clearly the pseudo true value will depend on the true DGP, namely $G$. Different distributions $G$ would yield different pseudo-true values for the \emph{same} likelihood $f$. Now, if we evaulate $H$ at the \emph{empirical} distribution $\widehat{G}_T$ we get the maximum likelihood estimator $\widehat{\theta}$ rather than the pseudo-true value $\theta_0$.

The influence function is in fact a \emph{functional derivative}. It allows us to evaulate, for example, how the pseudo-true value $\theta_0$  would change if we \emph{slightly} changed the distribution $G$ that generated the data by ``polluting'' it with a tiny mass point located at $y$. We could also consider how the maximum likelihood estimator, $\widehat{\theta}$, would change if we slightly changed the dataset, represented by empirical distribution function. 

Under certain regularity conditions,
	$$T(\widehat{G}_T) = T(G) + \frac{1}{T}\sum_{t=1}^T \mbox{infl}(G,Y_t) + o_p(T^{-1/2})$$
so that
	$$\sqrt{T}\left(H(\widehat{G}_T) - H(G) \right) \overset{d}{\rightarrow}N(0,\Omega)$$




\section{Mallow's $C_p$}
Suppose that we want to predict $y$ from $\mathbf{x}$ using a linear regression model:
	$$\underset{(T\times1)}{\textbf{y}} = \underset{(T\times K)}{X} \underset{(K\times 1)}{\beta} + \boldsymbol{\epsilon}$$
Where $E[\boldsymbol{\epsilon}|X] = 0$ and $Var(\boldsymbol{\epsilon}|X) = \sigma^2 \mathbf{I}$.  Some elements of $\beta$ might be zero or at least very small. We consider using only a subset of the regressors and ask whether this harms our predictions in a mean-squared error sense.  

Adopt the convention that $\widehat{\beta}_M$ has zeros for the elements of $\beta$ that are not estimated. In other words:
	$$X\widehat{\beta}_M = X_{(-M)}\textbf{0} + X_M (X_M'X_M)^{-1} X_M'\textbf{y} = P_M \textbf{y}$$

One idea would be to use the in-sample prediction error to compare models
	$$RSS(M) = (\textbf{y} - X\widehat{\beta}_M)'(\textbf{y} - X\widehat{\beta}_M)$$
But we know that this can never decrease as we add regressors. In constrast, out-of-sample predictive ability can easily decrease when we add more predictors: there's a bias-variance tradeoff that arises from estimation uncertainty.

Compare to the ``best we could possibly do'' using $X$. This is $X\beta$ but of course we don't know $\beta$.
\begin{eqnarray*}
	(X\widehat{\beta}_M - X\beta)'(X\widehat{\beta}_M - X\beta) &=& (P_M \mathbf{y} - X\beta)'(P_M \mathbf{y} - X\beta)\\
		&=&\left\{P_M(Y-X\beta) + (\mathbf{I} - P_M)X\beta \right\}' \left\{P_M(Y-X\beta) + (\mathbf{I} - P_M)X\beta  \right\}\\
		&=&\left\{ P_M \boldsymbol{\epsilon} + (\mathbf{I}- P_M)X\beta\right\}'\left\{ P_M \boldsymbol{\epsilon} + (\mathbf{I}- P_M)X\beta \right\}\\
		&=&\boldsymbol{\epsilon}'P_M'P_M \boldsymbol{\epsilon} + \beta'X'(\mathbf{I}-P_M)'P_M\boldsymbol{\epsilon} \\
			&&\quad \quad + \;\boldsymbol{\epsilon}'P_M'(\mathbf{I} - P_M)X\beta + \beta'X' (\mathbf{I} - P_M)(\mathbf{I} - P_M)X\beta\\
		&=& \boldsymbol{\epsilon}'P_M \boldsymbol{\epsilon} + \beta'X'(\mathbf{I} - P_M)X\beta
\end{eqnarray*}
where we have used the fact that $P_M$ and $(\mathbf{I}-P_M)$ are symmetric and idempotent and their product is zero. Thus,
	\begin{eqnarray*}
		\mbox{MSE}(M|X) &=& E\left[(X\widehat{\beta}_M - X\beta)'(X\widehat{\beta}_M - X\beta)|X \right]\\
		 &=& E\left[\boldsymbol{\epsilon}'P_M \boldsymbol{\epsilon}|X\right] + E\left[\beta'X'(\mathbf{I} - P_M)X\beta |X\right]\\
			&=&E\left[\mbox{trace}\left\{\boldsymbol{\epsilon}'P_M \boldsymbol{\epsilon}\right\}|X\right] + \beta'X'(\mathbf{I} - P_M)X\beta \\
		&=&\mbox{trace}\left\{E[\boldsymbol{\epsilon} \boldsymbol{\epsilon}'|X]P_M\right\} + \beta'X'(\mathbf{I} - P_M)X\beta \\
	&=&\mbox{trace}\left\{\sigma^2 P_M\right\} + \beta'X'(\mathbf{I} - P_M)X\beta \\
	&=& \sigma^2 k_M + \beta'X'(\mathbf{I} - P_M)X\beta
	\end{eqnarray*}
where $k_M$ is the number of regressors included in $X_M$ and we have used the fact that the trace of a projection matrix equals its dimension. 

The problem with the proceeding expression is that we don't know $\theta$ or $\sigma^2$ so we can't compare MSE across models. To get around this, we'll construct an unbiased estimator. Let $\widehat{\beta}$ be the estimator based on the full set of regressors, i.e.\ $\widehat{\beta} = (X'X)^{-1}X'\mathbf{y}$ and let $P_{X}$ be the corresponding projection matrix so that we have 
		$$X \widehat{\theta} = X(X'X)^{-1}X'\mathbf{y} = P_{X}\mathbf{y}$$
Then, using the fact that $P_MP_X = P_XP_M = P_M$, we have
	\begin{eqnarray*}
		E\left[\widehat{\theta}'X'(\mathbf{I} - P_M)X\widehat{\theta} |X\right] &=& E\left[\mathbf{y}'P_X'(\mathbf{I} - P_M)P_{X}\mathbf{y} |X\right]\\
			&=& E\left[\mathbf{y}'(P_X'P_X - P_X'P_MP_X)\mathbf{y} |X\right]\\
			&=& E\left[\mathbf{y}'(P_X - P_M)\mathbf{y} |X\right]\\
			&=& E\left[(X\beta + \boldsymbol{\epsilon})'(P_X - P_M)(X\beta + \boldsymbol{\epsilon})\mathbf{y} |X\right]\\
			&=& E\left[\beta'X(P_X - P_M)X\beta|X\right] + E[\boldsymbol{\epsilon}'(P_X - P_M)X\beta|X] \\
				&& \quad  + \; E[\beta'X'(P_X - P_M)\boldsymbol{\epsilon}|X] + E[\boldsymbol{\epsilon}'(P_X - P_M)\boldsymbol{\epsilon}|X]\\
			&=& \beta'X(P_X - P_M)X\beta +  E[\boldsymbol{\epsilon}'(P_X - P_M)\boldsymbol{\epsilon}|X]\\
			&=& \beta'X(P_X - P_M)X\beta +  E[\mbox{trace}\left\{\boldsymbol{\epsilon}'(P_X - P_M)\boldsymbol{\epsilon}\right\}|X]\\
			&=&\beta'X(P_X - P_M)X\beta +  \mbox{trace}\left\{E[\boldsymbol{\boldsymbol{\epsilon}\epsilon}'|X](P_X - P_M)\right\}\\
			&=&\beta'X(P_X - P_M)X\beta +  \mbox{trace}\left\{\sigma^2(P_X - P_M)\right\}\\ 
			&=&\beta'X(P_X - P_M)X\beta +  \sigma^2 \left(\mbox{trace}\left\{P_X\right\} - \mbox{trace}\left\{P_M\right\}\right)\\ 
			&=& \beta'X(P_X - P_M)X\beta +  \sigma^2 (K - k_M) 
	\end{eqnarray*}
Rearranging, and comparing to our expression from above, we have 
	$$E[\widehat{\theta}'X'(\mathbf{I} - P_M)X\widehat{\theta} +\sigma^2( 2k_M -  K)|X] = MSE(M|X)$$
We still don't know $\sigma^2$, but we can construct an unbiased estimate from the residuals of the full model as follows
	$$\widehat{\sigma}^2 = \frac{\mathbf{y}'(\mathbf{I} - P_X)\mathbf{y}}{(T-K)}$$
Now, as we calculated above,
	$$\widehat{\theta}'X'(\mathbf{I} - P_M)X\widehat{\theta} = \textbf{y}'(P_X - P_M)\textbf{y}$$
Hence, the following is an unbiased estimator of $\mbox{MSE}(M|X)$
	$$\mbox{MC}_M = \widehat{\theta}'X'(\mathbf{I} - P_M)X\widehat{\theta} +\widehat{\sigma}^2( 2k_M -  K)$$
To get Mallow's $C_p$ we rearrange this quantity. First, we have
\begin{eqnarray*}
	MC_M - 2\widehat{\sigma}^2k_M &=& \widehat{\theta}'X'(\mathbf{I} - P_M)X\widehat{\theta} - K\widehat{\sigma}^2\\
	&=& \mathbf{y}'P_X'(\mathbf{I} - P_M)P_X \mathbf{y}K\widehat{\sigma}^2\\
	&=&\mathbf{y}'(P_X'P_X - P_X'P_MP_X)\mathbf{y}K\widehat{\sigma}^2\\
	 &=& \mathbf{y}'(P_X - P_M)\mathbf{y} - \left(\frac{K}{T-K}\right)\mathbf{y}'\left(\mathbf{I} - P_X\right)\mathbf{y}\\
		&=& \left(\frac{T-K}{T-K}\right)(\mathbf{y}'P_X\mathbf{y} - \mathbf{y}'P_M\mathbf{y}) - \left(\frac{K}{T-K}\right)(\mathbf{y}'\mathbf{y}' - \mathbf{y}'P_X \mathbf{y})\\
			&=&\left(\frac{T-K}{T-K} + \frac{K}{T-K}\right)\mathbf{y}'P_X \mathbf{y}  - \left(\frac{T-K}{T-K}\right)\mathbf{y}'P_M\mathbf{y} - \left(\frac{K}{T-K}\right)\mathbf{y}'\mathbf{y} \\
			&=&\left(\frac{T}{T-K}\right)\mathbf{y}'P_X \mathbf{y} - \mathbf{y}'P_M\mathbf{y} - \left(\frac{K}{T-K} \right)\mathbf{y}'\mathbf{y}  \\
			&=& \left(\frac{T}{T-K}\right)\mathbf{y}'P_X \mathbf{y} - \mathbf{y}'P_M\mathbf{y} + \left(\frac{T - K - T}{T-K} \right)\mathbf{y}'\mathbf{y}  \\
			&=&  \left(\frac{T}{T-K} \right)\mathbf{y}'P_X \mathbf{y} - \mathbf{y}' P_M \mathbf{y} + \left(1 - \frac{T}{T-K}\right)\mathbf{y}'\mathbf{y} \\
			&=&  \left(\frac{T}{T-K} \right)\mathbf{y}'P_X \mathbf{y} - \mathbf{y}' P_M \mathbf{y} + \mathbf{y}'\mathbf{y} - \left(\frac{T}{T-K}\right)\mathbf{y}'\mathbf{y}  \\
			&=& \mathbf{y}'\mathbf{y} - \mathbf{y}'P_M\mathbf{y} - \left(\frac{T}{T-K} \right)(\mathbf{y}'\mathbf{y} - \mathbf{y}'P_m\mathbf{y})\\
			&=& \mathbf{y}'(\mathbf{I} - P_M)\mathbf{y} - \left(\frac{T}{T-K}\right)\mathbf{y}'\left(\mathbf{I} - P_X\right)\mathbf{y}\\
			&=& \mathbf{y}'(\mathbf{I} - P_M)\mathbf{y} - T\widehat{\sigma}^2\\
			&=& RSS(M) - T\widehat{\sigma^2}
\end{eqnarray*}
In other words,
	$$MC_M = RSS(M) + \widehat{\sigma}^2(2 k_M - T)$$
Dividing through by $\widehat{\sigma}^2$ gives Mallow's $C_p$
	$$C_p(M) = \frac{RSS(M)}{\widehat{\sigma}^2} + 2k_M - T$$

\section{Time Series Examples}
We won't go through all of the specifics here since they're almost identical to the material from above. Some more details can be found in
McQuarrie and Tsai (1998). The AR and VAR models are straightforward since, in the conditional formulation, they're just univariate and multivariate regression, respectively.

\subsection{Autoregressive Models}

\paragraph{Cross-Validation for AR}
The way we described it above, CV depended in independence. How can we adapt it for AR models? Roughly speaking, the idea is to use the fact that dependence dies out over time and treat observations that are ``far enough apart'' as \emph{approximately} independent. Specifically, we choose an integer value $h$ and assume that $y_t$ and $y_s$ can be treated as independent as long as $|s - t|>h$. This idea is called ``$h$-block cross-validation'' and was introduced by Burman, Chow \& Nolan (1994). As in the iid version of leave-one-out cross-validation, we still evaluate a loss function by predicting \emph{one} witheld observation at a time using a model estimated without it. The difference is that we also omit the $h$ neighboring observations \emph{on each side} when fitting the model. For example, if we choose to evaluate squared-error loss, the criterion is
	$$CV_h(1) = \frac{1}{T-p}\sum_{t = p+1}^T \left(y_t - \hat{y}_{(t)}^h\right)^2$$
where 
$$\hat{y}^h_{(t)} = \hat{\phi}^h_{1(t)} y_{t-1} + \hdots + \hat{\phi}^h_{1(t)}y_{t-p}$$
and $\hat{\phi}^h_{j(t)}$ denotes the $j$th parameter estimate from the conditional least-squares estimator with observations $y_{t-h}, \hdots,  y_{t+h}$ removed. We still have the question of what $h$ to choose. Here there is a trade-off between making the assumption of independence more plausible and leaving enough observations to get precise model estimates. Intriguingly, the simulation evidence presented in McQuarrie and Tsai (1998) suggests that setting $h=0$, which yields plain-vanilla leave-one-out CV, works well even in settings with dependence.

The idea of $h$-block cross-validation can also be adapted to versions of cross-validation other than leave-one-out. For details, see Racine (1997, 2000).



\subsection{Vector Autoregression Models}
Write without an intercept for simplicity (just demean everything)
	\begin{eqnarray*}
		\underset{(q\times 1)}{\textbf{y}_t} &=& \underset{(q\times q)}{\Phi_1} \textbf{y}_{t-1} + \hdots + \Phi_{p}\textbf{y}_{t-p} + \epsilon_t\\
		\boldsymbol{\epsilon}_t &\overset{iid}{\sim}& N_q(\mathbf{0}, \Sigma)
	\end{eqnarray*}
Conditional least squares estimation, sample size, etc.	
\begin{eqnarray*}
	FPE &=& \left| \widehat{\Sigma}_p \right| \left( \frac{T + qp}{T - qp}\right)^q\\ \\
	AIC &=& \log \left| \widehat{\Sigma}_p\right| + \frac{2pq^2 + q(q+1)}{T}\\ \\ 
	AIC_c &=& \log \left| \widehat{\Sigma}_p\right|  + \frac{(T + qp)q}{T - qp - q -1}\\ \\
	BIC &=& \log \left| \widehat{\Sigma}_p\right| +  \frac{\log(T)pq^2}{T}\\ \\ 
	HQ &=& \log \left| \widehat{\Sigma}_p\right| +  \frac{2 \log\log(T)pq^2}{T}
\end{eqnarray*}

\paragraph{Problems with VAR model selection}
	\begin{enumerate}
		\item If we fit $p$ lags, we lose $p$ observations under the conditional least squares estimation procedure.
		\item Adding a lag introduces $q^2$ additional parameters. 
	\end{enumerate}

\paragraph{Cross-Validation for VARs} In principle we could use the same $h$-block idea here as we did for for the AR example above. However, given the large number of parameters we need to estimate, the sample sizes witholding $2h+1$ observations at a time may be too small for this to work well. 




\subsection{Corrected AIC for State Space Models}
Problem with VARs and state space more generally is that we can easily have sample size small relative to number of parameters. In this case AIC-type criteria don't work well. Suggestions for simulation-based selection.

\paragraph{Cavanaugh \& Shumway (1997)} 

\paragraph{}
 


\section{Bayesian Information Criterion}
Since Frank2 talked about this in his part of the course, I won't discuss this derivation in class but I wanted to provide the details for completeness. As in our derivation of TIC and AIC, we'll consider a setting with an iid sample of scalar random variables $Y_1, \hdots, Y_T$. The results still hold in the more general case, but this simplifies the notation. 

\subsection{Overview of the BIC}
Despite its name, the BIC is \emph{not} a Bayesian procedure. It is a large-sample Frequentist \emph{approximation} to Bayesian model selection:
	\begin{enumerate}
		\item Begin with a uniform prior on the set of candidate models so that it suffices to maximize the Marginal Likelihood.
		\item The BIC is a large sample approximation to the Marginal Likelihood:
		$$\int \pi(\beta_i)f_i(\mathbf{y}|\beta_i)d\beta_i$$
		where $i$ indexes models $M_i$ in a set $\mathcal{M}$.
		\item As usual when Bayesian procedures are subjected to Frequentist asymptotics, the priors on parameters vanish in the limit.
		\item We proceed by a \emph{Laplace Approximation} to the Marginal Likelihood
	\end{enumerate}

\subsection{Laplace Approximation}
For the moment simplify the notation by suppressing dependence on $M_i$. We want to approximate: 
	$$\int \pi(\beta)f(\mathbf{y}|\beta)d\beta$$
This is actually a common problem in applications of Bayesian inference:
	\begin{itemize} 
		\item Notice that $\pi(\beta)f(\mathbf{y}|\beta)$ is the \emph{kernel} of some probability density, i.e.\ the density without its normalizing constant. 
		\item \emph{How do we know this?} By Bayes' Rule 
	$$\pi(\beta|\mathbf{y}) = \frac{\pi(\beta)f(\mathbf{y}|\beta)}{\int \pi(\beta)f(\mathbf{y}|\beta) d\beta }$$
is a proper probability density and the denominator is \emph{constant} with respect to $\beta$. (The parameter has been ``integrated out.'')
	\item In Bayesian inference, we specify $\pi(\beta)$ and $f(\mathbf{y}|\beta)$, so $\pi(\beta)f(\mathbf{y}|\beta)$ is known. But to calculate the posterior we need to \emph{integrate} to find the normalizing constant.
	\item Only in special cases (e.g.\ conjugate families) can we find the exact normalizing constant. Typically some kind of approximation is needed:  
		\begin{itemize}
			\item Importance Sampling
			\item Markov-Chain Monte Carlo (MCMC)
			\item \emph{Laplace Approximation}
		\end{itemize}
	\end{itemize}
The Laplace Approximation is an \emph{analytical approximation} based on Taylor Expansion arguments. In Bayesian applications, the expansion is carried out around the posterior mode, i.e.\ the mode of $\pi(\beta)f(\mathbf{y}|\beta)$, but we will expand around the Maximum likelihood estimator. 

\begin{pro}[Laplace Approximation]
	\label{pro:laplace}
$$\int \pi(\beta)f(\mathbf{y}|\beta)d\beta \approx \frac{\exp\left\{ \ell(\hat{\beta}) \right\} \pi(\hat{\beta})(2\pi)^{p/2}}{n^{p/2}\left| J(\hat{\beta}) \right|^{1/2}}$$
	Where $\hat{\beta}$ is the \emph{maximum likelihood estimator}, $p$ the dimension of $\beta$ and
		$$J(\hat{\beta}) = -\frac{1}{n} \frac{\partial^2 \log f(\mathbf{y}|\hat{\beta})}{\partial \beta \partial \beta'}$$
\end{pro}

\begin{proof}
A rigorous proof of this result is complicated. The following is a sketch. First write $\ell(\beta)$ for $\log{f(\mathbf{y}|\beta)}$ so that 
$$\pi(\beta)f(\mathbf{y}|\beta) = \pi(\beta) \exp{\left\{ \log{f(\mathbf{y}| \beta)} \right\}}=\pi(\beta) \exp{\left\{ \log{\ell(\beta)} \right\}}$$
By a second-order Taylor Expansion around the MLE $\hat{\beta}$
	\begin{equation}
	\label{taylorell}
		\ell(\beta) = \ell(\hat{\beta}) +\frac{1}{2} \left( \beta - \hat{\beta}  \right)' \frac{\partial^2 \ell(\hat{\beta})}{\partial \beta \partial \beta'} \left( \beta - \hat{\beta}  \right) + R_\ell
	\end{equation}
since the derivative of $\ell(\beta)$ is zero at $\hat{\beta}$ by the definition of MLE. A first-order expansion is sufficient for $\pi(\beta)$ because the derivative does not vanish at $\hat{\beta}$
	\begin{equation}
		\label{taylorpi}
		\pi(\beta) = \pi(\hat{\beta}) +  \frac{\partial \pi(\hat{\beta})}{\partial \beta'} \left(\beta - \hat{\beta}  \right)+ R_\pi
	\end{equation}
Substituting Equations \ref{taylorell} and \ref{taylorpi},
	\begin{eqnarray*}
		\int \pi(\beta)f(\mathbf{y}|\beta)d\beta &=& \int \exp\left\{ \ell(\hat{\beta}) +\frac{1}{2} \left( \beta - \hat{\beta}  \right)' \frac{\partial^2 \ell(\hat{\beta})}{\partial \beta \partial \beta'} \left( \beta - \hat{\beta}  \right) + R_\ell \right\}\\
		&&\;\;\;\;\;\;\times \left[ \pi(\hat{\beta}) + \left(\beta - \hat{\beta}  \right)' \frac{\partial \pi(\hat{\beta})}{\partial \beta} + R_\pi \right]  d\beta\\\\
		&=& \exp\left\{ \ell(\hat{\beta}) \right\} (I_1 + I_2 + I_3)
	\end{eqnarray*}
where
	\begin{eqnarray*}
		I_1 &=& \pi(\hat{\beta}) \int  \exp{\left\{ \frac{1}{2} \left( \beta - \hat{\beta}  \right)' \frac{\partial^2 \ell(\hat{\beta})}{\partial \beta \partial \beta'} \left( \beta - \hat{\beta}  \right) + R_\ell\right\}} d\beta\\
		I_2 &=& \frac{\partial \pi(\hat{\beta})}{\partial \beta'}   \int  \left(\beta - \hat{\beta}  \right) \exp{\left\{ \frac{1}{2} \left( \beta - \hat{\beta}  \right)' \frac{\partial^2 \ell(\hat{\beta})}{\partial \beta \partial \beta'} \left( \beta - \hat{\beta}  \right) + R_\ell\right\}} d\beta\\
\\
		I_3 &=& \int R_\pi \; \exp{\left\{ \frac{1}{2} \left( \beta - \hat{\beta}  \right)' \frac{\partial^2 \ell(\hat{\beta})}{\partial \beta \partial \beta'} \left( \beta - \hat{\beta}  \right) + R_\ell\right\}} d\beta
	\end{eqnarray*}
Under certain regularity conditions (not the standard ones!) we can treat $R_\ell$ and $R_\pi$ as approximately equal to zero for large $n$ uniformly in $\beta$, so that
	\begin{eqnarray*}
		I_1 &\approx& \pi(\hat{\beta}) \int  \exp{\left\{ \frac{1}{2} \left( \beta - \hat{\beta}  \right)' \frac{\partial^2 \ell(\hat{\beta})}{\partial \beta \partial \beta'} \left( \beta - \hat{\beta}  \right) \right\}} d\beta\\
		I_2 &\approx& \frac{\partial \pi(\hat{\beta})}{\partial \beta'}   \int  \left(\beta - \hat{\beta}  \right) \exp{\left\{ \frac{1}{2} \left( \beta - \hat{\beta}  \right)' \frac{\partial^2 \ell(\hat{\beta})}{\partial \beta \partial \beta'} \left( \beta - \hat{\beta}  \right) \right\}} d\beta\\
		I_3 &\approx& 0
	\end{eqnarray*}	
Because $\hat{\beta}$ is the MLE, 
	$$\frac{\partial^2 \ell(\hat{\beta})}{\partial \beta \partial \beta'}$$ 
must be negative definite, so 
	$$-\frac{\partial^2 \ell(\hat{\beta})}{\partial \beta \partial \beta'}$$
is positive definite. It follows that 
	$$ \exp{\left\{ \frac{1}{2} \left( \beta - \hat{\beta}  \right)' \frac{\partial^2 \ell(\hat{\beta})}{\partial \beta \partial \beta'} \left( \beta - \hat{\beta}  \right) \right\}} =  \exp{\left\{ -\frac{1}{2} \left( \beta - \hat{\beta}  \right)'\left[ \left(-\frac{\partial^2 \ell(\hat{\beta})}{\partial \beta \partial \beta'}\right)^{-1} \right]^{-1}\left( \beta - \hat{\beta}  \right) \right\}}$$
can be viewed as the kernel of a Normal distribution with mean $\hat{\beta}$ and variance matrix 
	$$\left(-\frac{\partial^2 \ell(\hat{\beta})}{\partial \beta \partial \beta'}\right)^{-1}$$
Thus,
	$$\int  \exp{\left\{ \frac{1}{2} \left( \beta - \hat{\beta}  \right)' \frac{\partial^2 \ell(\hat{\beta})}{\partial \beta \partial \beta'} \left( \beta - \hat{\beta}  \right) \right\}} d\beta = \left(2\pi\right)^{p/2}\left| \left(-\frac{\partial^2 \ell(\hat{\beta})}{\partial \beta \partial \beta'}\right)^{-1} \right|^{1/2}$$
and
	$$\int \left(\beta - \hat{\beta}  \right)  \exp{\left\{ \frac{1}{2} \left( \beta - \hat{\beta}  \right)' \frac{\partial^2 \ell(\hat{\beta})}{\partial \beta \partial \beta'} \left( \beta - \hat{\beta}  \right) \right\}} d\beta = 0$$
Therefore,
	\begin{eqnarray*}
		\int \pi(\beta)f(\mathbf{y}|\beta)d\beta &\approx& \exp\left\{ \ell(\hat{\beta}) \right\}\pi(\hat{\beta}) \left(2\pi\right)^{p/2}\left| \left(-\frac{\partial^2 \ell(\hat{\beta})}{\partial \beta \partial \beta'}\right)^{-1} \right|^{1/2}\\
		&=&  \exp\left\{ \ell(\hat{\beta}) \right\}\pi(\hat{\beta}) \left(2\pi\right)^{p/2}\left|n \left(-\frac{1}{n}\frac{\partial^2 \ell(\hat{\beta})}{\partial \beta \partial \beta'}\right) \right|^{-1/2}\\\\
		&=&\frac{ \exp\left\{ \ell(\hat{\beta}) \right\}\pi(\hat{\beta}) \left(2\pi\right)^{p/2}}{n^{p/2}\left| J(\hat{\beta}) \right|^{1/2}}
	\end{eqnarray*}
\end{proof}


\subsection{Finally the BIC}
Now we re-introduce the dependence on the model $M_i$. Taking logs of the Laplace Approximation and multiplying by two (again, this is traditional but has no effect on model comparisons)
	\begin{eqnarray*}
		2 \log f(y|M_i) &=& 2 \log \left\{ \int f_i(y|\beta_i)\pi(\beta_i)d\beta_i \right\}\\
		&\approx& 2\ell(\hat{\beta}_i) -p\log(n) + p \log(2\pi)- \pi(\hat{\beta}_i)-\log \left| J(\hat{\beta_i}) \right|
	\end{eqnarray*}
The first two terms are $O_p(n)$ and $O_p(\log{n})$, while the last three are $O_p(1)$, hence negligible as $n\rightarrow \infty$. This gives us Schwarz's BIC
	$$BIC(M_i) = 2\log{f_i(\mathbf{y}|\hat{\beta}_i)} - p\log{n}$$
We choose the model $M_i$ for which $BIC(M_i)$ is largest. Notice that the prior on the parameter, $\pi(\beta)$, drops out in the limit, and recall that we began by putting a uniform prior on the \emph{models} under consideration. 


\end{document}