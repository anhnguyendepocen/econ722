
\documentclass[12pt]{article}
\usepackage[margin=1.5in]{geometry}
\usepackage{todonotes}
\usepackage{amssymb,amsmath,amsthm,graphicx}
\usepackage{rotating}
\usepackage{setspace}
\usepackage{enumerate}
\usepackage{graphicx}
\usepackage{listings}
\usepackage{multirow}
\usepackage{color}
\usepackage{threeparttable}
\usepackage{caption}
\usepackage{subcaption}
\usepackage{hyperref}

\newtheorem{assump}{Assumption}[section]
\newtheorem{pro}{Proposition}[section]
\newtheorem{lem}{Lemma}[section]
\newtheorem{thm}{Theorem}[section]
\newtheorem{cor}{Corollary}[section]
\newtheorem{ineq}{Inequality}[section]
\newtheorem{defn}{Definition}[section]
\newtheorem{rem}{Remark}[section]
\newtheorem{ex}{Example}[section]
\theoremstyle{definition}
\newtheorem{alg}{Algorithm}[section]


\linespread{1.3}

\begin{document}

\title{Lecture 7: High-Dimensional Linear Regression}

\author{Francis J.\ DiTraglia}

\maketitle 

\section{Introduction}
So far we've looked at model selection. For example, we considered the problem of choosing the ``best'' set of regressors for a forecasting problem. Here, the idea was to consider dropping regressors with small coefficients to get a favorable bias-variance tradeoff. There are several problems with this approach. First, variable selection can be unstable because of the discrete nature of the problem: small changes in the underlying data could lead to large changes in the selected set of regressors. Second, it is only computationally infeasible to consider all possible subsets of regressors when $p < 30$. Our colleague Andy Postelwaite actually has a microeconomic theory paper about this called ``Fact Free Learning.'' You should check it out: it's very interesting!

In this lecture we'll consider an alternative to model selection called ``shrinkage.'' The idea is roughly as follows: rather than making a discrete choice of which variables are ``in'' and which are ``out,'' it might make more sense to leave everything in the model but ``regularize'' or ``shrink'' the estimated coefficients away from the maximum likelihood estimator, much as a Bayesian prior does. Rather than attempting to incorporate prior beliefs, however, here the idea is merely to find a clever way of adding bias that buys us a large decrease in variance. There will still be a model selection component here, but it will involve a single, continuous ``tuning'' or ``smoothing'' parameter.

\section{Review of Matrix Decompositions}
\subsection{The QR Decomposition}
Any $n\times k$ matrix $A$ with full column rank can be decomposed as $A = QR$, where $R$ is an $k\times k$ upper triangular matrix and $Q$ is an $n\times k$ matrix with orthonormal columns. The columns of $A$ are \emph{orthogonalized} in $Q$ via the Gram-Schmidt process. Since $Q$ has orthogonal columns, we have $Q'Q = I_k$. It is \emph{not} in general true that $QQ' = I$, however. In the special case where $A$ is square, $Q^{-1} = Q'$.

\paragraph{Note:} The way we have defined things here is here is sometimes called the ``thin'' or ``economical'' form of the QR decomposition, e.g.\ \texttt{qr\_econ} in Armadillo. In our ``thin'' version, $Q$ is an $n\times k$ matrix with orthogonal columns. In the ``thick'' version, $Q$ is an $n\times n$ \emph{orthogonal} matrix. Let $A = QR$ be the ``thick'' version and $A = Q_1 R_1$ be the ``thin'' version. The connection between the two is as follows:
  $$A = QR = Q \left[\begin{array}
    {c} R_1 \\ 0 
  \end{array} \right] = \left[  \begin{array}
    {cc} Q_1 & Q_2
  \end{array}\right]\left[\begin{array}
    {c} R_1 \\ 0 
  \end{array} \right] = Q_1 R_1$$

\paragraph{Least-Squares via the QR Decomposition} We can calculate the least squares estimator of $\beta$ as follows
\begin{eqnarray*}
  \widehat{\beta} &=& (X'X)^{-1} X'y = \left[(QR)' (QR) \right]^{-1} (QR)' y\\
    &=&\left[ R' Q' Q R\right]^{-1} R'Q' y = (R'R)^{-1} R'Q y\\
    &=& R^{-1} (R')^{-1} R' Q'y = R^{-1} Q'y
\end{eqnarray*}
In other words, $\widehat{\beta}$ is the solution to $R\beta = Q'y$. While it may not be immediately apparent, this is a much easier system to solve that the normal equations $(X'X) \beta = X'y$. Because $R$ is \emph{upper triangular} we can solve $R\beta = Q'y$ extremely quickly. The product $Q'y$ is a vector, call it $v$, so the system is simply
  $$\left[
    \begin{array}
      {cccccc}
      r_{11} & r_{12}  & r_{13}& \cdots & r_{1,n-1} & r_{1k} \\
      0 & r_{22} & r_{23}&\cdots & r_{2,n-1} & r_{2k}\\
      0&  0 &  r_{33}& \cdots & r_{3,n-1} & r_{3k}\\  
      \vdots & \vdots & \ddots& \ddots & \vdots & \vdots\\
      0 & 0 & \cdots &0  & r_{k-1, k-1} & r_{k-1, k} \\
      0 & 0 & \cdots & 0 & 0 & r_{k}
    \end{array}
  \right] \left[ \begin{array}
    {ccc}
    \beta_1 \\ \beta_2 \\ \beta_3 \\ \vdots \\ \beta_{k-1} \\ \beta_k
  \end{array}\right] = \left[ \begin{array}
    {c}
    v_1  \\ v_2  \\ v_3 \\  \vdots \\ v_{k-1} \\ v_{k}
  \end{array}\right]
  $$
Hence, $\beta_k = v_k / r_k$ which we can substitute into $\beta_{k-1} r_{k-1,k-1} + \beta_k r_{k-1,k} = v_{k-1}$ to solve for $\beta_{k-1}$, and so on. This is called \textbf{back substitution}. We can use the same idea when a matrix is \emph{lower triangular} only in reverse: this is called \textbf{forward substitution}.

To calculate the variance matrix $\sigma^2 (X'X)^{-1}$ for the least-squares estimator, simply note from the derivation above that $(X'X)^{-1} = R^{-1} (R^{-1})'$ . Inverting $R$, however, is easy: we simply apply back-substitution \emph{repeatedly}. Let $A$ be the inverse of $R$, $\mathbf{a}_j$ be the $j$th column of $A$, and $\mathbf{e}_j$ be the $j$th element of the $k\times k$ identity matrix, i.e.\ the $j$th standard basis vector. Inverting $R$ is equivalent to solving $R \mathbf{a}_1 = \mathbf{e}_1$, followed by $R \mathbf{a}_2 = \mathbf{e}_2$, and so on all the way up to $R \mathbf{a}_k = \mathbf{e}_k$. In Armadillo, if you enclose a matrix in \texttt{trimatu()} or \texttt{trimatl()}, and then request the inverse, the library will carry out backward or forward substitution, respectively.

\paragraph{Othogonal Projection Matrices and the QR Decomposition}
Consider a projection matrix $P_X = X (X'X)^{-1}X'$. Provided that $X$ has full column rank, we have
begin
  $$P_X  = QR(R'R)^{-1}R'Q' = QRR^{-1} (R')^{-1}R'Q' = QQ'$$
Recall that, in general, it is \emph{not} true that $QQ' = I$ even though $Q'Q = I$ because we're using the \emph{economical} QR decomposition in which $Q$ has orthonormal columns but may not be a square matrix. Just to make this completely transparent, consider a very simple example:
	$$X = \left[ \begin{array}
		{cc} 1 & 0 \\ 0 & 1 \\ 0 & 0
	\end{array}\right]$$
Then, we have
	$$X'X = \left[\begin{array}
		{ccc} 1& 0 & 0 \\ 0 & 1 & 0
	\end{array} \right]\left[ \begin{array}
		{cc} 1 & 0 \\ 0 & 1 \\ 0 & 0
	\end{array}\right] = \left[\begin{array}
		{cc} 1 & 0 \\ 0 & 1
	\end{array} \right]$$
but 
	$$XX' = \left[ \begin{array}
		{cc} 1 & 0 \\ 0 & 1 \\ 0 & 0
	\end{array}\right]\left[\begin{array}
		{ccc} 1& 0 & 0 \\ 0 & 1 & 0
	\end{array} \right] = \left[\begin{array}
		{ccc} 1 & 0 & 0 \\ 0 & 1 & 0 \\ 0 & 0 & 0
	\end{array} \right]$$
It's important to keep the fact that $UU' \neq I$ in mind when using the QR decomposition for more complicated matrix calculations.

\subsection{The Singular Value Decomposition}
The Singular Value Decomposition (SVD) is probably the most elegant result in linear algebra. It's also an invaluable computational and theoretical tool in statistics and econometrics. I can only give a brief overview here, but I'd encourage you to learn more when you have time. Some excellent references are Strang (1993) and Kalman (2002).

\paragraph{The SVD} Any $m \times n$ matrix $A$ of arbitrary rank $r$ can be decomposed according to 
	$$X = UDV' = (\mbox{orthogonal})(\mbox{diagonal})(\mbox{orthogonal})$$
	\begin{itemize}
	 	\item $U$ is an $m\times m$ orthogonal matrix whose columns contain the eigenvectors of $AA'$
	 	\item $V$ is an $n\times n$ orthogonal matrix whose columns contain the eigenvectors of $A'A$
	 	\item $D$ is an $m\times n$ matrix whose first $r$ main diagonal elements are the  \emph{singular values} $d_1, \hdots, d_r$. All other elements of $D$ are zero.
	 	\item The singular values $d_1, \hdots, d_r$ are the positive eigenvalues of $A'A$ which are \emph{identical} to the positive eigenvalues of $AA'$.
	 \end{itemize} 

\paragraph{The Four Fundamental Subspaces} It turns out that the SVD provides orthonormal bases for each of the so-called ``fundamental subspaces'' of a matrix $A$. In particular:
	\begin{enumerate}
		\item \textbf{column space}: first $r$ columns of $U$
		\item \textbf{row space}: first $r$ columns of $V$
		\item \textbf{null space}: last $n - r$ columns of $V$
		\item \textbf{left null space}: last $m - r$ columns of $U$
	\end{enumerate}

\paragraph{SVD for Symmetric Matrices} If $A$ is symmetric then, by the spectral theorem, we can write $A = Q\Lambda Q'$ where $\Lambda$ is a diagonal matrix containing the eigenvalues of $A$ and $Q$ is an orthonormal matrix whose columns are the corresponding eigenvectors. In this case, $U = V = Q$ and $D$ is simply the absolute value of $\Lambda$ (i.e.\ negative eigenvalues become positive singular values).

\paragraph{SVD for Positive Definite Matrices} If $A$ is not only symmetric but \emph{positive definite}, then $A = Q\Lambda Q'$ is the \emph{same decomposition} as $A = UDV'$: $U=V=Q$ and $\Lambda = D$.

\paragraph{The ``Economical'' SVD}
The number of singular values equals $r$, the rank of $A$, which is at most $\max\{m,n\}$. This means that some of the columns of $U$ or $V$ will be \emph{irrelevant} since they will be multiplied by zeros in $D$. Accordingly, most linear algebra libraries provide an ``economical'' SVD that only calculate the columns of $U$ and $V$ that are multiplied by non-zero values in $D$. In Armadillo, for example, the command is \texttt{svd\_econ}. 

We can write the economical SVD in summation form as 
	$$A = \sum_{i=1}^r d_i \textbf{u}_i \textbf{v}_i'$$
where $r = \mbox{rank}(A)$ and the singular values $d_i$ are arranged in order from largest to smallest. In matrix form, this is given by:
	$$\underset{(n\times p)}{A} = \underset{(n\times r)}{U} \underset{(r\times r)}{D} \underset{(r\times p)}{V'}$$
In the economical SVD, $U$ and $V$ may no longer be square, so they are not orthogonal matrices but their \emph{columns} are still orthonormal.

\paragraph{Approximation Property of SVD} The Frobenius norm of a matrix $A$ is given by
	$$||A||_F = \sqrt{\sum_{i=1}^m \sum_{i=1}^n a_{ij}^2} = \sqrt{\mbox{trace}(A'A)}$$
Using this norm as a measure of ``approximation error'', it can be shown that the SVD provides the \emph{best low rank approximation} to a matrix $X$. 

Using the ``economical'' form of the SVD, we can write
	$$X = \sum_{i=1}^r d_i \textbf{u}_i \textbf{v}_i'$$
where the index is $i$ is defined such that the \emph{largest} singular value comes first, followed by the second largest, and so on. This expression gives the rank-$r$ matrix $X$ as a \emph{sum} of $r$ rank-1 matrices. Now suppose that the rank of $X$ is large and we wanted to \emph{approximate} $X$ using a matrix $\widehat{X}_L$ with rank $L<k$. If we measure the reconstruction error using the Frobenius norm, it can be shown that the \emph{truncated SVD}
	$$\widehat{X}_L = \sum_{i=1}^{L} d_i \textbf{u}_i \textbf{v}_i'$$ 	
provides the best rank $L$ approximation to $X$. In other words, $\widehat{X}_L$ is the $\arg \min$  over all rank $L$ matrices of the quantity $||X - \widehat{X}_L||_F$. It is also possible to provide bounds on the quality of the approximation, and thus choose an appropriate truncation. 

\section{Gauss-Markov, meet James-Stein}
Consider the linear regression model
	$$\mathbf{y} = X\beta + \boldsymbol{\epsilon}$$
In Econ 705 you learned that ordinary least squares (OLS) is the minimum variance unbiased linear estimator of $\beta$ under the assumptions $E[\boldsymbol{\epsilon}|X] = \mathbf{0}$ and $Var(\mathbf{\epsilon}|X) = \sigma^2 I$. When the second assumption fails, you learned that generalized least squares (GLS) provides a lower variance estimator than OLS. All of this is fine, as far as it goes, but there's an obvious objection: why are we restricting ourselves to unbiased estimators? Generically, we know that there is a bias-variance tradeoff. So what happens if we allow ourselves to consider biased estimators? 



\subsection{Dominance and Admissibility}
To understand what follows, we'll need two concepts from decision theory: \textbf{dominance} and \emph{admissibility}. Let $\widehat{\theta}$ and $\widetilde{\theta}$ be two estimators of $\theta$, and $R$ be a risk function, e.g.\ MSE. We say that $\widehat{\theta}$ \textbf{dominates} $\widetilde{\theta}$ with respect to $R$ if $R(\widehat{\theta},\theta) \leq R(\widetilde{\theta},\theta)$ for \emph{all} possible values of $\theta$ wand the inequality is \emph{strict} for \emph{at least one} possible value of $\theta$. We say that $\widehat{\theta}$ is \textbf{admissible} if there is no estimator that dominates it. To prove that an estimator is \textbf{inadmissible} it suffices to find an estimator that dominates it.

\subsection{The James-Stein Estimator}
It turns out that OLS is inadmissible relative to MSE loss when $p\geq 3$. This result was so surprising when first discovered that it is sometimes called ``Stein's Paradox.'' Efron and Morris (1977) provide a nice discussion. An estimator that can be shown to dominate OLS is the so-called ``positive-part James-Stein estimator'' which is given by
	$$\widehat{\beta}^{JS} = \widehat{\beta}\left[1 -\frac{(p-2)\sigma^2}{\widehat{\beta}' \widehat{\beta}} \right]_+$$
where $\widehat{\beta}$ is the OLS estimator and $(x)_+ = \max(x,0)$. To make this operational, we substitute an estimator of $\sigma^2$, for example one based on the OLS residuals. We see that this estimator shrinks us \emph{away} from the least-squares fit and towards zero. 

Since James-Stein beats OLS, we know that shrinkage is a good idea. Now we'll consider some more general forms of shrinkage, starting with Ridge Regression.



\section{Ridge Regression} 
Ridge regression is a technique that was originally designed to address the problem of multicollinearity. When two or more predictors are very strongly correlated, OLS can become unstable. For example, if $x_1$ and $x_2$ are \emph{nearly} linearly dependent, a large positive coefficient $\beta_1$ could effectively \emph{cancel out} a large negative coefficient $\beta_2$. Ridge Regression attempts to solve this problem by \emph{shrinking the estimated coefficients towards zero and towards each other}. This is accomplished by adding a squared $L_2$-norm ``penalty'' to the OLS objective function, yielding
	$$\widehat{\beta}_{Ridge} =\underset{\beta}{\arg \min} (\mathbf{y} - \textbf{1}_n\beta_0 - X\beta)' (\mathbf{y} - \textbf{1}_n \beta_0 - X\beta) + \lambda \beta'\beta$$
where $\textbf{1}_n$ is an $(n\times 1)$ vector of ones, $\beta_0$ denotes the regression intercept and $\beta = (\beta_1, \hdots, \beta_p)'$ the remaining coefficients. Note that we do \emph{not} penalize the intercept in Ridge Regression. The easiest and most common way to handle this is simply to de-mean both $X$ and $\mathbf{y}$ before proceeding. so that there is no intercept and the problem becomes
	$$\widehat{\beta}_{Ridge} =\underset{\beta}{\arg \min}(\mathbf{y} - X\beta)' (\mathbf{y} - X\beta) + \lambda \beta'\beta$$
Throughout these notes we will assume that everything has been de-meaned so there is no intercept.

\subsection{Ridge is \emph{Not} Scale Invariant} When we carry out OLS, if we re-scale a regressor $\mathbf{x}$, replacing it with $c \mathbf{x}$ where $c$ is some nonzero constant, then the corresponding OLS coefficient estimate is scaled by $1/c$ to compensate. In other words, OLS is \emph{scale invariant}. The same is not true of Ridge Regression, so it is common to convert the columns of the design matrix to the same units before proceeding. The usual way of handling this is simply to \emph{standardize} each of the regressors.

\subsection{Another Way to Express Ridge Regression}
The following is an equivalent statement of the Ridge Regression problem:
$$\widehat{\beta}_{Ridge} = \underset{\beta}{\arg \min}(\mathbf{y} - X\beta)' (\mathbf{y} - X\beta) \quad \mbox{subject to} \quad \beta'\beta \leq t$$
In other words, Ridge Regression is like least squares ``on a budget.'' If you want to make one coefficient estimate larger, you have to make another one smaller. The ``income'' level $t$ maps one-to-one to $\lambda$, although the mapping is data-dependent.

\subsection{Ridge as Bayesian Linear Regression}
As you may recall from the first part of the semester, Bayesian models with informative priors automatically provide a form of shrinkage. Indeed, many frequentist shrinkage estimators can be expressed in Bayesian terms. Provided that we ignore the regression constant, the solution to Ridge Regression is \emph{equivalent} to MAP (maximum a posteriori) estimation based on the following  Bayesian regression model
	\begin{eqnarray*}
		y|X, \beta, \sigma^2 &\sim& N(X\beta,\sigma^2 I_n) \\
		\beta &\sim& N(\mathbf{0}, \tau^2 I_p)
	\end{eqnarray*}
where $\sigma^2$ is assumed known and $\lambda = \sigma^2/\tau^2$. In other words, Ridge Regression gives the \textbf{posterior mode}. Since this model is conjugate, the posterior is normal. Thus, in addition to being the MAP estimator, the solution to Ridge Regression is also the posterior mean.


\subsection{An Explicit Formula for Ridge Regression}
The objective function is
\begin{eqnarray*}
	Q_{ridge}&=& (\mathbf{y} - X\beta)' (\mathbf{y} - X\beta) + \lambda \beta'\beta\\
	&=&\mathbf{y}'\mathbf{y} - \beta'X \mathbf{y} - \mathbf{y}'X\beta + \beta'X'X \beta + \lambda \beta' I_p \beta\\
	&=& \mathbf{y}'\mathbf{y} - 2 \mathbf{y}'X\beta + \beta'(X'X + \lambda I_p)\beta
\end{eqnarray*}
Recall the following facts about matrix differentiation\footnote{See, for example, Harville (1997; Chapter 15).}
	\begin{eqnarray*}
		\frac{\partial (\mathbf{a}' \mathbf{x})}{\partial \mathbf{x}}  &=& \mathbf{a}\\
		\frac{\partial( \mathbf{x}'A \mathbf{x})}{\partial \mathbf{x}} &=& (A + A')\mathbf{x}
	\end{eqnarray*}
Thus, we have 
$$\frac{\partial}{\partial \beta} Q(\beta) = -2X'\bf{y} + 2(X'X + \lambda I_p)\beta$$
since $(X'X + \lambda I_p)$ is symmetric. Thus, the first order condition is 
	$$X'\textbf{y} = (X'X + \lambda I_p)\beta$$
Hence,
	$$\widehat{\beta}_{Ridge} = (X'X + \lambda I_p)^{-1} X'\textbf{y}$$
So is $(X'X + \lambda I_p)$ guaranteed to be invertible? We need this to be the case for the solution of the Ridge Regression problem to be unique. In the following section, we'll provide an alternative way of analyzing the problem by turning it into something we're more familiar with: OLS.

\subsection{Ridge Regression via OLS}
From the first half of the semester, you may recall that Bayesian linear regression can be thought of as ``plain-vanilla'' OLS using a design matrix that has been \emph{augemented} with ``fake'' observations that represent the prior. This turns out to be a very helpful way of looking at Ridge Regression. Define

$$\widetilde{\textbf{y}} = \left[ \begin{array}
	{c} \textbf{y} \\ \textbf{0}_p
\end{array}\right], \quad \quad \widetilde{X} = \left[ \begin{array}
	{c} X \\ \sqrt{\lambda} I_p
\end{array}\right]$$
The objective function for Ridge Rgression is \emph{identical} to the OLS objective function for the augmented dataset, namely
	$$\underset{\beta}{\arg \min} \left(\widetilde{\mathbf{y}} - \widetilde{X}\beta\right)'\left(\widetilde{\mathbf{y}} - \widetilde{X}\beta\right)$$
Which we can show as follows:
\begin{eqnarray*}
	\left(\widetilde{\mathbf{y}} - \widetilde{X}\beta\right)'\left(\widetilde{\mathbf{y}} - \widetilde{X}\beta\right) &=& \left[\begin{array}
		{cc} (\mathbf{y} - X\beta)' & (-\sqrt{\lambda}\beta)'
	\end{array} \right] \left[\begin{array}
		{c} (\mathbf{y} - X\beta) \\ -\sqrt{\lambda} \beta
	\end{array} \right]\\
		&=& (\mathbf{y} - X\beta)' (\mathbf{y} - X\beta) + \lambda \beta'\beta\\
\end{eqnarray*}

\subsection{Ridge is Always Unique} We know that the OLS estimator is only unique provided that the design matrix has full column rank. In constrast there is \emph{always} a unique solution to the Ridge Regression problem, even when there are more regressors than observations. This follows \emph{immediately} from the preceding: the columns of $\sqrt{\lambda}I_p$ are linearly independent, so the columns of the augmented data matrix $\widetilde{X}$ are \emph{also} linearly independent, \emph{regardless} of whether the same holds for the columns of $X$. Thus we can use Ridge Regression even in settings in which there are more regressors than observations!

\subsection{Efficient Calculations for Ridge Regression}
Since we've reduced Ridge Regression to OLS on a modified dataset, we can use the QR decomposition for efficient and stable calculations. First take the QR decomposition of $\widetilde{X}$, namely $\widetilde{X} = QR$. Then, 
	$$\widehat{\beta}_{Ridge} = (\widetilde{X}' \widetilde{X})^{-1} \widetilde{X}'\,\widetilde{\textbf{y}} = R^{-1} Q' \,\widetilde{\textbf{y}}$$
which we can obtain by back-solving the system $R\widehat{\beta}_{Ridge} = Q'\, \widetilde{\mathbf{y}}$. In situations where $p \gg n$, it's actually much faster to use the SVD rather than the QR decomposition because the rank of $X$ will be $n$ in this case. For details on how to implement this, see Murphy (2012; Section 7.5.2).



\subsection{Predictive Bias and Variance of OLS and Ridge}

Take the \emph{economical} singular value decomposition of the $(n\times p)$ centered (and possibly standardized) design matrix $X$. We have
	$$\underset{(n\times p)}{X} = \underset{(n\times r)}{U} \underset{(r\times r)}{D} \underset{(r\times p)}{V'}$$
where $r = \mbox{rank}(X)$ and thus
	$$X'X = (UDV')'(UDV') = VDU'UDV' = VD^2V'$$
Provided that the columns of $X$ are linearly indepdendent, $r = p$ and hence $VD^2 V'$ is the \emph{eigen-decomposition} of $X'X$. Since $X$ is centered, the sample covariance matrix of the regressors is $S = X'X/n$. Since it is simply a scalar multiple of $X'X$, the sample covariance matrix $S$ has the \emph{same eigenvectors} as $X'X$, namely the columns of $V$.


Continuing under the assumption that $r = p$ so that $V$ is $(p\times p)$ and $V'V = V'V = I_p$, we have
	\begin{eqnarray*}
		\widehat{\beta}_{Ridge} &=& \left(X'X  +\lambda I_p\right)^{-1} X'\mathbf{y} \\
		&=& \left(VD^2V' + \lambda I_p\right)^{-1}(UDV')' \mathbf{y}\\
		&=& \left(VD^2V' + \lambda VV'\right)^{-1}VDU' \mathbf{y}\\
		&=& \left[V(D^2 + \lambda I_p)V'\right]^{-1}VDU'\mathbf{y}\\
		&=&(V')^{-1}(D^2 + \lambda I_p)^{-1}V^{-1}VDU' \mathbf{y}\\
		&=& V(D^2 + \lambda I_p)^{-1}DU'\mathbf{y}\\
		&=& \left[\sum_{i=1}^p \mathbf{v}_i \left(\frac{d_i}{d_i^2 + \lambda}\right) \mathbf{u}_i'\right] \mathbf{y}
	\end{eqnarray*}

\begin{eqnarray*}
	\widehat{y}_{Ridge} &=& X\widehat{\beta}_{Ridge} = (UDV')V(D^2 + \lambda I_p)^{-1}DU'\mathbf{y}\\
	&=& UD(D^2 + \lambda I_p)^{-1}DU'\mathbf{y} = UD^2(D^2 + \lambda I_p)^{-1}U'\mathbf{y}\\
	&=&\left[\sum_{i=1}^p \mathbf{u}_i \left(\frac{d_i^2}{d_i^2 +\lambda}\right)\mathbf{u}_i' \right] \mathbf{y}
\end{eqnarray*}
Can we relate this to PCR now? Yes! the principal components are the eigenvectors of the sample covariance matrix of $X$. These are the \emph{same} as the eigenvectors of $X'X$ which are simply the columns of $V$, namely the $\mathbf{v}_i$.

This is a linear smoother! We can use our trick for leave-one-out cross-validation here.

\subsection{Choosing $\lambda$}
To implement Ridge Regression we need a method of choosing $\lambda$. One idea is cross-validation, either k-fold or leave-one-out. Since Ridge Regression is a linear smoother, we can use the computational trick you derived on Problem Set 5 to avoid directly calculating the leave-one-out estimators. The role of the OLS ``hat matrix'' $H = X(X'X)^{-1}X'$ is played by $H(\lambda) = X(X'X + \lambda I_p)^{-1} X'$.

Degrees of freedom, compare to OLS. AIC, BIC, cross-validation, k-fold.

Degrees of freedom. For OLS there are $p$ free parameters and this is also the trace of the hat matrix:
	$$\mbox{trace}(H) = \mbox{trace}\left\{ X(X'X)^{-1}X'\right\} = \mbox{trace}\left\{X'X(X'X)^{-1} \right\} = \mbox{trace}\left\{ I_p\right\}=p$$
Use this analogy for Ridge:
	\begin{eqnarray*}
		\mbox{df}(\lambda) &=& \mbox{trace}\left\{ H(\lambda)\right\} =  \mbox{trace}\left\{ X(X'X + \lambda I_p)^{-1} X'\right\}\\
		&=&\mbox{trace}\left\{UD^2(D^2 + \lambda I_p)^{-1}U'\right\}\\
		&=& \mbox{trace}\left\{U'UD^2(D^2 + \lambda I_p)^{-1}\right\}\\
		&=& \mbox{trace}\left\{D^2(D^2 + \lambda I_p)^{-1}\right\}\\
		&=& \sum_{i=1}^p \frac{d_i^2}{d_i^2 + \lambda}
	\end{eqnarray*}
so we see that the degrees of freedom tend to zero as $\lambda \rightarrow \infty$, and equal $p$ if $\lambda = 0$, which is OLS. We're assuming here that $X$ has full rank. 


\section{Principal Components Regression}

\section{LASSO}

\end{document}