
\documentclass[12pt]{article}
\usepackage[margin=1.5in]{geometry}
\usepackage{todonotes}
\usepackage{amssymb,amsmath,amsthm,graphicx}
\usepackage{rotating}
\usepackage{setspace}
\usepackage{enumerate}
\usepackage{graphicx}
\usepackage{listings}
\usepackage{multirow}
\usepackage{color}
\usepackage{threeparttable}
\usepackage{caption}
\usepackage{subcaption}
\usepackage{hyperref}

\newtheorem{assump}{Assumption}[section]
\newtheorem{pro}{Proposition}[section]
\newtheorem{lem}{Lemma}[section]
\newtheorem{thm}{Theorem}[section]
\newtheorem{cor}{Corollary}[section]
\newtheorem{ineq}{Inequality}[section]
\newtheorem{defn}{Definition}[section]
\newtheorem{rem}{Remark}[section]
\newtheorem{ex}{Example}[section]
\theoremstyle{definition}
\newtheorem{alg}{Algorithm}[section]


\linespread{1.3}

\begin{document}

\title{Lecture 7: High-Dimensional Linear Regression}

\author{Francis J.\ DiTraglia}

\maketitle 

\section{Review of Matrix Decompositions}


\subsection{The QR Decomposition}
Any $n\times k$ matrix $A$ with full column rank can be decomposed as $A = QR$, where $R$ is an $k\times k$ upper triangular matrix and $Q$ is an $n\times k$ matrix with orthonormal columns. The columns of $A$ are \emph{orthogonalized} in $Q$ via the Gram-Schmidt process. Since $Q$ has orthogonal columns, we have $Q'Q = I_k$. It is \emph{not} in general true that $QQ' = I$, however. In the special case where $A$ is square, $Q^{-1} = Q'$.

\paragraph{Note:} The way we have defined things here is here is sometimes called the ``thin'' or ``economical'' form of the QR decomposition, e.g.\ \texttt{qr\_econ} in Armadillo. In our ``thin'' version, $Q$ is an $n\times k$ matrix with orthogonal columns. In the ``thick'' version, $Q$ is an $n\times n$ \emph{orthogonal} matrix. Let $A = QR$ be the ``thick'' version and $A = Q_1 R_1$ be the ``thin'' version. The connection between the two is as follows:
  $$A = QR = Q \left[\begin{array}
    {c} R_1 \\ 0 
  \end{array} \right] = \left[  \begin{array}
    {cc} Q_1 & Q_2
  \end{array}\right]\left[\begin{array}
    {c} R_1 \\ 0 
  \end{array} \right] = Q_1 R_1$$

\paragraph{Least-Squares via the QR Decomposition} We can calculate the least squares estimator of $\beta$ as follows
\begin{eqnarray*}
  \widehat{\beta} &=& (X'X)^{-1} X'y = \left[(QR)' (QR) \right]^{-1} (QR)' y\\
    &=&\left[ R' Q' Q R\right]^{-1} R'Q' y = (R'R)^{-1} R'Q y\\
    &=& R^{-1} (R')^{-1} R' Q'y = R^{-1} Q'y
\end{eqnarray*}
In other words, $\widehat{\beta}$ is the solution to $R\beta = Q'y$. While it may not be immediately apparent, this is a much easier system to solve that the normal equations $(X'X) \beta = X'y$. Because $R$ is \emph{upper triangular} we can solve $R\beta = Q'y$ extremely quickly. The product $Q'y$ is a vector, call it $v$, so the system is simply
  $$\left[
    \begin{array}
      {cccccc}
      r_{11} & r_{12}  & r_{13}& \cdots & r_{1,n-1} & r_{1k} \\
      0 & r_{22} & r_{23}&\cdots & r_{2,n-1} & r_{2k}\\
      0&  0 &  r_{33}& \cdots & r_{3,n-1} & r_{3k}\\  
      \vdots & \vdots & \ddots& \ddots & \vdots & \vdots\\
      0 & 0 & \cdots &0  & r_{k-1, k-1} & r_{k-1, k} \\
      0 & 0 & \cdots & 0 & 0 & r_{k}
    \end{array}
  \right] \left[ \begin{array}
    {ccc}
    \beta_1 \\ \beta_2 \\ \beta_3 \\ \vdots \\ \beta_{k-1} \\ \beta_k
  \end{array}\right] = \left[ \begin{array}
    {c}
    v_1  \\ v_2  \\ v_3 \\  \vdots \\ v_{k-1} \\ v_{k}
  \end{array}\right]
  $$
Hence, $\beta_k = v_k / r_k$ which we can substitute into $\beta_{k-1} r_{k-1,k-1} + \beta_k r_{k-1,k} = v_{k-1}$ to solve for $\beta_{k-1}$, and so on. This is called \textbf{back substitution}. We can use the same idea when a matrix is \emph{lower triangular} only in reverse: this is called \textbf{forward substitution}.

To calculate the variance matrix $\sigma^2 (X'X)^{-1}$ for the least-squares estimator, simply note from the derivation above that $(X'X)^{-1} = R^{-1} (R^{-1})'$ . Inverting $R$, however, is easy: we simply apply back-substitution \emph{repeatedly}. Let $A$ be the inverse of $R$, $\mathbf{a}_j$ be the $j$th column of $A$, and $\mathbf{e}_j$ be the $j$th element of the $k\times k$ identity matrix, i.e.\ the $j$th standard basis vector. Inverting $R$ is equivalent to solving $R \mathbf{a}_1 = \mathbf{e}_1$, followed by $R \mathbf{a}_2 = \mathbf{e}_2$, and so on all the way up to $R \mathbf{a}_k = \mathbf{e}_k$. In Armadillo, if you enclose a matrix in \texttt{trimatu()} or \texttt{trimatl()}, and then request the inverse, the library will carry out backward or forward substitution, respectively.

\paragraph{Othogonal Projection Matrices and the QR Decomposition}
Consider a projection matrix $P_X = X (X'X)^{-1}X'$. Provided that $X$ has full column rank, we have
begin
  $$P_X  = QR(R'R)^{-1}R'Q' = QRR^{-1} (R')^{-1}R'Q' = QQ'$$
Recall that, in general, it is \emph{not} true that $QQ' = I$ even though $Q'Q = I$ because we're using the \emph{economical} QR decomposition in which $Q$ has orthonormal columns but may not be a square matrix. Just to make this completely transparent, consider a very simple example:
	$$X = \left[ \begin{array}
		{cc} 1 & 0 \\ 0 & 1 \\ 0 & 0
	\end{array}\right]$$
Then, we have
	$$X'X = \left[\begin{array}
		{ccc} 1& 0 & 0 \\ 0 & 1 & 0
	\end{array} \right]\left[ \begin{array}
		{cc} 1 & 0 \\ 0 & 1 \\ 0 & 0
	\end{array}\right] = \left[\begin{array}
		{cc} 1 & 0 \\ 0 & 1
	\end{array} \right]$$
but 
	$$XX' = \left[ \begin{array}
		{cc} 1 & 0 \\ 0 & 1 \\ 0 & 0
	\end{array}\right]\left[\begin{array}
		{ccc} 1& 0 & 0 \\ 0 & 1 & 0
	\end{array} \right] = \left[\begin{array}
		{ccc} 1 & 0 & 0 \\ 0 & 1 & 0 \\ 0 & 0 & 0
	\end{array} \right]$$
It's important to keep the fact that $UU' \neq I$ in mind when using the QR decomposition for more complicated matrix calculations.

\subsection{The Singular Value Decomposition}
The Singular Value Decomposition (SVD) is probably the most elegant result in linear algebra. It's also an invaluable computational and theoretical tool in statistics and econometrics. I can only give a brief overview here, but I'd encourage you to learn more when you have time. Some excellent references are Strang (1993) and Kalman (2002).

\paragraph{The SVD} Any $m \times n$ matrix $A$ of arbitrary rank $r$ can be decomposed according to 
	$$X = UDV' = (\mbox{orthogonal})(\mbox{diagonal})(\mbox{orthogonal})$$
	\begin{itemize}
	 	\item $U$ is an $m\times m$ orthogonal matrix whose columns contain the eigenvectors of $AA'$
	 	\item $V$ is an $n\times n$ orthogonal matrix whose columns contain the eigenvectors of $A'A$
	 	\item $D$ is an $m\times n$ matrix whose first $r$ main diagonal elements are the  \emph{singular values} $d_1, \hdots, d_r$. All other elements of $D$ are zero.
	 	\item The singular values $d_1, \hdots, d_r$ are the positive eigenvalues of $A'A$ which are \emph{identical} to the positive eigenvalues of $AA'$.
	 \end{itemize} 

\paragraph{The Four Fundamental Subspaces} It turns out that the SVD provides orthonormal bases for each of the so-called ``fundamental subspaces'' of a matrix $A$. In particular:
	\begin{enumerate}
		\item \textbf{column space}: first $r$ columns of $U$
		\item \textbf{row space}: first $r$ columns of $V$
		\item \textbf{null space}: last $n - r$ columns of $V$
		\item \textbf{left null space}: last $m - r$ columns of $U$
	\end{enumerate}

\paragraph{SVD for Symmetric Matrices} If $A$ is symmetric then, by the spectral theorem, we can write $A = Q\Lambda Q'$ where $\Lambda$ is a diagonal matrix containing the eigenvalues of $A$ and $Q$ is an orthonormal matrix whose columns are the corresponding eigenvectors. In this case, $U = V = Q$ and $D$ is simply the absolute value of $\Lambda$ (i.e.\ negative eigenvalues become positive singular values).

\paragraph{SVD for Positive Definite Matrices} If $A$ is not only symmetric but \emph{positive definite}, then $A = Q\Lambda Q'$ is the \emph{same decomposition} as $A = UDV'$: $U=V=Q$ and $\Lambda = D$.

\paragraph{The ``Economical'' SVD}
The number of singular values equals $r$, the rank of $A$, which is at most $\max\{m,n\}$. This means that some of the columns of $U$ or $V$ will be \emph{irrelevant} since they will be multiplied by zeros in $D$. Accordingly, most linear algebra libraries provide an ``economical'' SVD that only calculate the columns of $U$ and $V$ that are multiplied by non-zero values in $D$. In Armadillo, for example, the command is \texttt{svd\_econ}. 

We can write the economical SVD in summation form as 
	$$A = \sum_{i=1}^r d_i \textbf{u}_i \textbf{v}_i'$$
where $r = \mbox{rank}(A)$ and the singular values $d_i$ are arranged in order from largest to smallest. In matrix form, this is given by:
	$$\underset{(n\times p)}{A} = \underset{(n\times r)}{U} \underset{(r\times r)}{D} \underset{(r\times p)}{V'}$$
In the economical SVD, $U$ and $V$ may no longer be square, so they are not orthogonal matrices but their \emph{columns} are still orthonormal.

\paragraph{Approximation Property of SVD} The Frobenius norm of a matrix $A$ is given by
	$$||A||_F = \sqrt{\sum_{i=1}^m \sum_{i=1}^n a_{ij}^2} = \sqrt{\mbox{trace}(A'A)}$$
Using this norm as a measure of ``approximation error'', it can be shown that the SVD provides the \emph{best low rank approximation} to a matrix $X$. 

Using the ``economical'' form of the SVD, we can write
	$$X = \sum_{i=1}^r d_i \textbf{u}_i \textbf{v}_i'$$
where the index is $i$ is defined such that the \emph{largest} singular value comes first, followed by the second largest, and so on. This expression gives the rank-$r$ matrix $X$ as a \emph{sum} of $r$ rank-1 matrices. Now suppose that the rank of $X$ is large and we wanted to \emph{approximate} $X$ using a matrix $\widehat{X}_L$ with rank $L<k$. If we measure the reconstruction error using the Frobenius norm, it can be shown that the \emph{truncated SVD}
	$$\widehat{X}_L = \sum_{i=1}^{L} d_i \textbf{u}_i \textbf{v}_i'$$ 	
provides the best rank $L$ approximation to $X$. In other words, $\widehat{X}_L$ is the $\arg \min$  over all rank $L$ matrices of the quantity $||X - \widehat{X}_L||_F$. It is also possible to provide bounds on the quality of the approximation, and thus choose an appropriate truncation. 

\section{Gauss-Markov, meet James-Stein}
Consider the linear regression model
	$$\mathbf{y} = X\beta + \boldsymbol{\epsilon}$$
In Econ 705 you learned that ordinary least squares (OLS) is the minimum variance unbiased linear estimator of $\beta$ under the assumptions $E[\boldsymbol{\epsilon}|X] = \mathbf{0}$ and $Var(\mathbf{\epsilon}|X) = \sigma^2 I$. When the second assumption fails, you learned that generalized least squares (GLS) provides a lower variance estimator than OLS. All of this is fine, as far as it goes, but there's an obvious objection: why are we restricting ourselves to unbiased estimators? Generically, we know that there is a bias-variance tradeoff. So what happens if we allow ourselves to consider biased estimators? Does some form of the Gauss-Markov Theorem still hold?



\paragraph{Admissibility}

\subsection{The James-Stein Estimator}

\subsection{The Positive-Part James-Stein Estimator}

\section{Ridge Regression} 
Ridge regression is a technique that was originally designed to address the problem of multicollinearity. When two or more predictors are very strongly correlated, OLS can become unstable. For example, if $x_1$ and $x_2$ are \emph{nearly} linearly dependent, a large positive coefficient $\beta_1$ could effectively \emph{cancel out} a large negative coefficient $\beta_2$. Ridge Regression attempts to solve this problem by \emph{shrinking the estimated coefficients towards zero and towards each other}. This is accomplished by adding a squared $L_2$-norm ``penalty'' to the OLS objective function, yielding
	$$\widehat{\beta}_{Ridge} = (\mathbf{y} - \textbf{1}_n\beta_0 - X\beta)' (\mathbf{y} - \textbf{1}_n \beta_0 - X\beta) + \lambda \beta'\beta$$
where $\textbf{1}_n$ is an $(n\times 1)$ vector of ones, $\beta_0$ denotes the regression intercept and $\beta = (\beta_1, \hdots, \beta_p)'$ the remaining coefficients. Note that we do \emph{not} penalize the intercept in Ridge Regression. The easiest and most common way to handle this is simply to de-mean both $X$ and $\mathbf{y}$ before proceeding. so that there is no intercept and the problem becomes
	$$\widehat{\beta}_{Ridge} =(\mathbf{y} - X\beta)' (\mathbf{y} - X\beta) + \lambda \beta'\beta$$
Throughout these notes we will assume that everything has been de-meaned so there is no intercept.


\paragraph{Ridge is \emph{Not} Scale Invariant}


\subsection{Ridge as Bayesian Linear Regression}
As you may recall from the first part of the semester, Bayesian models with informative priors automatically provide a form of shrinkage. Indeed, many frequentist shrinkage estimators can be expressed in Bayesian terms. Provided that we ignore the regression constant, the solution to Ridge Regression is \emph{equivalent} to MAP (maximum a posteriori) estimation based on the following  Bayesian regression model
	\begin{eqnarray*}
		y|\mathbf{x}, \beta, \sigma^2 &\sim& N(\mathbf{x}'\beta|\sigma^2) \\
		\beta &\sim& N_p(\mathbf{0}, \tau^2 I_p)
	\end{eqnarray*}
where $\sigma^2$ is assumed known and $\lambda = \sigma^2/\tau^2$. In other words, Ridge Regression gives the \textbf{posterior mode}. Since this model is conjugate, the posterior is normal. Thus, in addition to being the MAP estimator, the solution to Ridge Regression is also the posterior mean.

\subsection{Another Way to Express Ridge Regression}
Data-dependent mapping.

\subsection{Ridge Regression via OLS}
From the first half of the semester, you may recall that Bayesian linear regression can be thought of as ``plain-vanilla'' OLS using a design matrix that has been \emph{augemented} with ``fake'' observations that represent the prior. This turns out to be a very helpful way of looking at Ridge Regression. Define

$$\widetilde{\textbf{y}} = \left[ \begin{array}
	{c} \textbf{y} \\ \textbf{0}_p
\end{array}\right], \quad \quad \widetilde{X} = \left[ \begin{array}
	{c} X \\ \sqrt{\lambda} I_p
\end{array}\right]$$
The objective function for Ridge Rgression is \emph{identical} to the OLS objective function for the augmented dataset, namely
	$$\underset{\beta}{\arg \min} \left(\widetilde{\mathbf{y}} - \widetilde{X}\beta\right)'\left(\widetilde{\mathbf{y}} - \widetilde{X}\beta\right)$$
Which we can show as follows:
\begin{eqnarray*}
	\left(\widetilde{\mathbf{y}} - \widetilde{X}\beta\right)'\left(\widetilde{\mathbf{y}} - \widetilde{X}\beta\right) &=& \left[\begin{array}
		{cc} (\mathbf{y} - X\beta)' & (-\sqrt{\lambda}\beta)'
	\end{array} \right] \left[\begin{array}
		{c} (\mathbf{y} - X\beta) \\ -\sqrt{\lambda} \beta
	\end{array} \right]\\
		&=& (\mathbf{y} - X\beta)' (\mathbf{y} - X\beta) + \lambda \beta'\beta\\
\end{eqnarray*}

\subsection{Ridge is Always Unique} We know that the OLS estimator is only unique provided that the design matrix has full column rank. In constrast there is \emph{always} a unique solution to the Ridge Regression problem, even when there are more regressors than observations. This follows \emph{immediately} from the preceding: the columns of $\sqrt{\lambda}I_p$ are linearly independent, so the columns of the augmented data matrix $\widetilde{X}$ are \emph{also} linearly independent, \emph{regardless} of whether the same holds for the columns of $X$. Thus we can use Ridge Regression even in settings in which there are more regressors than observations!

\subsection{Efficient Calculations for Ridge Regression}

\paragraph{Calculations When $p\gg n$} 


\subsection{Predictive Bias and Variance of OLS and Ridge}

Take the \emph{economical} singular value decomposition of the $(n\times p)$ centered (and possibly standardized) design matrix $X$. We have
	$$\underset{(n\times p)}{X} = \underset{(n\times r)}{U} \underset{(r\times r)}{D} \underset{(r\times p)}{V'}$$
where $r = \mbox{rank}(X)$ and thus
	$$X'X = (UDV')'(UDV') = VDU'UDV' = VD^2V'$$
Provided that the columns of $X$ are linearly indepdendent, $r = p$ and hence $VD^2 V'$ is the \emph{eigen-decomposition} of $X'X$. Since $X$ is centered, the sample covariance matrix of the regressors is $S = X'X/n$. Since it is simply a scalar multiple of $X'X$, the sample covariance matrix $S$ has the \emph{same eigenvectors} as $X'X$, namely the columns of $V$.


Continuing under the assumption that $r = p$ so that $V$ is $(p\times p)$ and $V'V = V'V = I_p$, we have
	\begin{eqnarray*}
		\widehat{\beta}_{Ridge} &=& \left(X'X  +\lambda I_p\right)^{-1} X'\mathbf{y} \\
		&=& \left(VD^2V' + \lambda I_p\right)(UDV')' \mathbf{y}\\
		&=& \left(VD^2V' + \lambda VV'\right)VDU' \mathbf{y}\\
		&=& V(D^2 + \lambda I_p)V'VDU'\mathbf{y}\\
		&=&V(D^2 + \lambda I_p)DU' \mathbf{y}\\
		&=& \sum_{i=1}^p \mathbf{v}
	\end{eqnarray*}

\subsection{Choosing $\lambda$}
Degrees of freedom, compare to OLS. AIC, BIC, cross-validation, k-fold.

\section{Principal Components Regression}

\section{LASSO}

\end{document}