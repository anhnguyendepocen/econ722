%%%%%%%%%%%%%%%%%%%%%%%%%%%%%%%%%%%%%%%%%%%%%%
\section{QR Decomposition}
%%%%%%%%%%%%%%%%%%%%%%%%%%%%%%%%%%%%%%%%%%%%%%
\begin{frame}
  \frametitle{QR Decomposition}

\begin{block}{Result}
Any $n\times k$ matrix $A$ with full column rank can be decomposed as $A = QR$, where $R$ is an $k\times k$ upper triangular matrix and $Q$ is an $n\times k$ matrix with orthonormal columns. 
\end{block}

\begin{block}{Notes}
  \begin{itemize}  
  \item Columns of $A$ are \emph{orthogonalized} in $Q$ via Gram-Schmidt. 
 \item Since $Q$ has orthogonal columns, $Q'Q = I_k$. 
 \item It is \emph{not} in general true that $QQ' = I$.
 \item If $A$ is square, then $Q^{-1} = Q'$.
\end{itemize}
\end{block}
\end{frame}
%%%%%%%%%%%%%%%%%%%%%%%%%%%%%%%%%%%%%%%%%%%%%%
\begin{frame}
  \frametitle{Different Conventions for the QR Decomposition}
  

  \begin{block}{Thin aka Economical QR}
    $Q$ is an $n\times k$ with orthonormal columns ( \texttt{qr\_econ} in Armadillo).
  \end{block}

  \begin{block}{Thick QR}
    $Q$ is an $n\times n$ \emph{orthogonal} matrix.
  \end{block}

  \begin{block}{Relationship between Thick and Thin}
    
Let $A = QR$ be the ``thick'' QR and $A = Q_1 R_1$ be the ``thin'' QR: 
  $$A = QR = Q \left[\begin{array}
    {c} R_1 \\ 0 
  \end{array} \right] = \left[  \begin{array}
    {cc} Q_1 & Q_2
  \end{array}\right]\left[\begin{array}
    {c} R_1 \\ 0 
  \end{array} \right] = Q_1 R_1$$
  \end{block}

  \alert{My preferred convention is the thin QR\dots}

\end{frame}
%%%%%%%%%%%%%%%%%%%%%%%%%%%%%%%%%%%%%%%%%%%%%%
\begin{frame}
  \frametitle{Least Squares via QR Decomposition}
  \begin{block}{Let $X = QR$}
\begin{eqnarray*}
  \widehat{\beta} &=& (X'X)^{-1} X'y = \left[(QR)' (QR) \right]^{-1} (QR)' y\\
    &=&\left[ R' Q' Q R\right]^{-1} R'Q' y = (R'R)^{-1} R'Q y\\
    &=& R^{-1} (R')^{-1} R' Q'y = R^{-1} Q'y
\end{eqnarray*}

In other words, $\widehat{\beta}$ solves $R\beta = Q'y$.
\end{block}

\begin{block}{Why Bother?}
  Much easier and faster to solve $R\beta = Q'y$ than the normal equations $(X'X)\beta = X'y$ since $R$ is \alert{upper triangular}.
\end{block}
\end{frame}
%%%%%%%%%%%%%%%%%%%%%%%%%%%%%%%%%%%%%%%%%%%%%%
\begin{frame}
  \frametitle{Back-Substitution to Solve $R\beta = Q'y$}

The product $Q'y$ is a vector, call it $v$, so the system is simply
  $$\left[
    \begin{array}
      {cccccc}
      r_{11} & r_{12}  & r_{13}& \cdots & r_{1,n-1} & r_{1k} \\
      0 & r_{22} & r_{23}&\cdots & r_{2,n-1} & r_{2k}\\
      0&  0 &  r_{33}& \cdots & r_{3,n-1} & r_{3k}\\  
      \vdots & \vdots & \ddots& \ddots & \vdots & \vdots\\
      0 & 0 & \cdots &0  & r_{k-1, k-1} & r_{k-1, k} \\
      0 & 0 & \cdots & 0 & 0 & r_{k}
    \end{array}
  \right] \left[ \begin{array}
    {ccc}
    \beta_1 \\ \beta_2 \\ \beta_3 \\ \vdots \\ \beta_{k-1} \\ \beta_k
  \end{array}\right] = \left[ \begin{array}
    {c}
    v_1  \\ v_2  \\ v_3 \\  \vdots \\ v_{k-1} \\ v_{k}
  \end{array}\right]
  $$
$\beta_k = v_k / r_k \Rightarrow$ substitute this into $\beta_{k-1} r_{k-1,k-1} + \beta_k r_{k-1,k} = v_{k-1}$ to solve for $\beta_{k-1}$, and so on.
\end{frame}
%%%%%%%%%%%%%%%%%%%%%%%%%%%%%%%%%%%%%%%%%%%%%%
\begin{frame}
  \frametitle{Calculating the Least Squares Variance Matrix $\sigma^2 (X'X)^{-1}$}
  \begin{itemize}
    \item Since $X = QR$, $(X'X)^{-1} = R^{-1} (R^{-1})'$
    \item Easy to invert $R$: just apply \alert{repeated} back-substitution:
      \begin{itemize}
        \item Let $A = R^{-1}$ and $\mathbf{a}_j$ be the $j$th column of $A$.
        \item Let $\mathbf{e}_j$ be the $j$th standard basis vector.
        \item Inverting $R$ is equivalent to solving $R \mathbf{a}_1 = \mathbf{e}_1$, followed by $R \mathbf{a}_2 = \mathbf{e}_2$, \dots, $R \mathbf{a}_k = \mathbf{e}_k$.
      \end{itemize}
    \item If you enclose a matrix in \texttt{trimatu()} or \texttt{trimatl()}, and request the inverse $\Rightarrow$ Armadillo will carry out backward or forward substitution, respectively.
  \end{itemize}

\end{frame}
%%%%%%%%%%%%%%%%%%%%%%%%%%%%%%%%%%%%%%%%%%%%%%
\begin{frame}
  \frametitle{QR Decomposition for Orthogonal Projections}

Let $X$ have full column rank and define $P_X = X (X'X)^{-1}X'$
  $$P_X  = QR(R'R)^{-1}R'Q' = QRR^{-1} (R')^{-1}R'Q' = QQ'$$

  It is \emph{not} in general true that $QQ' = I$ even though $Q'Q = I$ since $Q$ need not be square in the economical QR decomposition. 

\end{frame}
%%%%%%%%%%%%%%%%%%%%%%%%%%%%%%%%%%%%%%%%%%%%%%
\section{Singular Value Decomposition}
%%%%%%%%%%%%%%%%%%%%%%%%%%%%%%%%%%%%%%%%%%%%%%
\begin{frame}
  \frametitle{The Singular Value Decomposition (SVD)}
  
Any $m \times n$ matrix $A$ of arbitrary rank $r$ can be written 
	$$X = UDV' = (\mbox{orthogonal})(\mbox{diagonal})(\mbox{orthogonal})$$
  \vspace{-2em}
	\begin{itemize}
	 	\item $U$ = $m\times m$ orthog.\ matrix whose cols contain e-vectors of $AA'$
	 	\item $V$ = $n\times n$ orthog.\ matrix whose cols contain e-vectors of $A'A$
	 	\item $D$ = $m\times n$ matrix whose first $r$ main diagonal elements are the  \emph{singular values} $d_1, \hdots, d_r$. All other elements are zero.
	 	\item The singular values $d_1, \hdots, d_r$ are the positive eigenvalues of $A'A$ which are \emph{identical} to the positive eigenvalues of $AA'$.
    \item (E-values of $A'A$ and $AA'$ could be zero but not negative)
	 \end{itemize} 
\end{frame}
%%%%%%%%%%%%%%%%%%%%%%%%%%%%%%%%%%%%%%%%%%%%%%
\begin{frame}
  \frametitle{SVD for Symmetric Matrices}
  
If $A$ is \textbf{symmetric} then $A = Q\Lambda Q'$ where $\Lambda$ is a diagonal matrix containing the e-values of $A$ and $Q$ is an orthonormal matrix whose columns are the corresponding e-vectors. Accordingly:

  \[
    AA' = (Q\Lambda Q')(Q\Lambda Q')' = Q \Lambda Q' Q \Lambda Q' = Q \Lambda^2 Q'
  \]
  and similarly
  \[
    A'A = (Q\Lambda Q')'(Q\Lambda Q') = Q \Lambda Q' Q \Lambda Q' = Q \Lambda^2 Q'
  \]
  using the fact that $Q$ is orthogonal and $\Lambda$ diagonal.
  Thus, when $A$ is symmetric the SVD reduces to $U = V = Q$ and $D =  \Lambda^2$ so that \emph{negative} eigenvalues become \emph{positive} singular values.
\end{frame}

%%%%%%%%%%%%%%%%%%%%%%%%%%%%%%%%%%%%%%%%%%%%%%
\begin{frame}
  \frametitle{The Economical SVD}
\begin{itemize}
  \item Number of singular values is $r = \mbox{Rank}(A) \leq \max\{m, n\}$
  \item Some cols of $U$ or $V$ multiplied by zeros in $D$
  \item Economical SVD: only keep columns in $U$ and $V$ that are multiplied by non-zeros in $D$
  (Armadillo: \texttt{svd\_econ})
  \item Summation form: $\displaystyle A = \sum_{i=1}^r d_i \textbf{u}_i \textbf{v}_i'$ where $d_1 \leq d_2 \leq \dots \leq d_r$
\item Matrix form: $\displaystyle\underset{(n\times p)}{A} = \underset{(n\times r)}{U} \underset{(r\times r)}{D} \underset{(r\times p)}{V'}$
\end{itemize}

\vspace{1em}
\alert{In the economical SVD, $U$ and $V$ may no longer be square, so they are not orthogonal matrices but their \emph{columns} are still orthonormal.}
\end{frame}
%%%%%%%%%%%%%%%%%%%%%%%%%%%%%%%%%%%%%%%%%%%%%%

